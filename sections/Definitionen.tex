

\chapter{Definitionen}
\label{ch:Definitionen}



\section{Definition: Einwegverpackung - im Zusammenhang mit
Bauprodukten und Baustoffen}
\label{sec:Definitionen:Einwegverpackung im Zusammenhang mit Bauprodukten und Baustoffen}

Mehrere Definitionen möglich. Daraus meine eigene Definition aufbauen und darauf referezieren.

%(https://packpart.eu/glossar/einwegverpackungen/)
-Eine Einwegverpackung ist eine Verpackung, die für den !einmaligen Gebrauch! bestimmt ist. Diese Verpackung dient oft dem Verkauf eines bestimmten Produktes und ist aufgrund der !Stabilität sowie aus anderen Gesichtspunkten wie Hygiene und Komfort nicht für die Wiederverwendung geeignet!. Nach !einmaligem Gebrauch muss diese Verpackung entsorgt oder dem Wertstoffkreislauf zugeführt werden!.(Einwegverpackung – Definition)

-Die Verpackung hat einen erheblichen Einfluss auf die Höhe der Logistikkosten und die Effektivität von Logistikprozessen(Logistik)
-Einwegverpackungen haben im Überseetransport gegenüber den Mehrwegverpackungen deutliche Vorteile. Das liegt daran, dass sich die Abmessungen von Bauteilen oft ändern und eine Anpassung oder Neuentwicklung einer Mehrwegverpackung unrentabel wäre.(Logistik)
-Verpackungen aus Kunststoff sind wiederverwertbar und gelangen über das Recycling wieder zurück in den Stoffkreislauf, in dem sie als Einweg- oder Mehrwegverpackung wiederbenutzt werden können(Rohstoffe2)

-Die !Rohstoffe! für ihre Herstellung können Pappe, Papier, Kunststoffe wie PET, Polyethylen und andere Arten von Polymeren und nachwachsenden, biologisch abbaubaren Rohstoffen sein.Kunststoffe, Papier, Bambus und Holz. Glas oder Kunststoff

    -!Ansprüchen! an bedarfsgerechte und sichere Lagerung sowie Transport
    -langzeitstabile, hygienische und für Kunden attraktive Verwahrung
    -Ladegut sicher und kostengünstig das Ziel erreichen.
    -hoher Oberflächengüte



Zur eindeutigen Festlegung von wichtigen Begriffen ist eine Definiion häufig verwendeter Begriffe notwendig. Sie bildet zusammen mit dem Kapitel "Abgrenzung der Arbeit" ein Fundament, auf das in dieser Arbeit referenziert wird.
Dazu gehören die Begriffe Einwegverpackung sowie Mehrwegverpackung, als auch der Begriff Packmittel, der als übergeordneter Begriff jegliche Art von Verpackungen abdeckt. Hinzu kommen die beiden Begriffe Bauprodukte sowie Baustoffe.   




%Thema: Minimierung von Plastikabfällen auch mein Thema?
%Thema: biologisch abbaubare Verpackungen als Teil meiner BA?
%-->(https://packpart.eu/glossar/einwegverpackungen/(Ökologische Einwegverpackungslösungen))




\section{Definition: Mehrwegverpackung - im Zusammenhang mit
Bauprodukten und Baustoffen}
\label{sec:Definitionen:Mehrwegverpackung im Zusammenhang mit Bauprodukten und Baustoffen}

(%https://packpart.eu/glossar/einwegverpackungen/)
-Im Gegensatz zu Einwegverpackungen bestehen Mehrwegverpackungen aus langlebigen und wiederverwendbaren Materialien, die im Alltag oder in der Produktion mehrfach verwendet werden können.(Einwegverpackung – Definition)


%(https://packpart.eu/glossar/mehrwegverpackung/)
Definition: Mehrwegverpackung: 
-Verpackung, die zum mehrmaligen Gebrauch bestimmt ist. Wiederverwendung nach Reinigungs- und Aufbereitungsprozess. Der Lebenszyklus beträgt oft mehrere Jahre. 
-Unterschied zur Einwegverpackung: hauptsächlich die Dauer des Lebenszyklus(aufgrund stabilierer Verarbeitung)


-Begriff Umlauf:(wie häufig etwas wiederverwendet werden kann-->Indikator)
-Mehrwegverpackung Beispiele
    Gitterboxen
    Europaletten -->Beladung, Auslage und Vertrieb
    Kunststoffboxen
    Mehrwegflaschen
    Getränkekisten
    Joghurtgläser
-Umsetzung und Organisation eines Mehrwegsystems(-->Indikator)
-Gesetzliche Vorgaben-->Indikator
-umweltfreundliches Image -->Indikator
-Begriff: Tertiärverpackung(Transport und Lagerung)
-Lagerung der Mehrwegverpackungen(zb Stapeln)-->Indikator
-Mehrwegverpackungen sind besonders gut für sperrige und schwere Packgüter


-Kriterien für die Wahl eines Verpackungsmaterials:(v.a. für den optimalen Schutz) %(https://packpart.eu/glossar/packmittel/)
    -Transport: Bei der Wahl des Packmittels ist sicherzustellen, dass die Produkte auf dem Transportweg dank des Packmittels gut geschützt werden. Zum Beispiel muss per Luftfracht transportierte Ware atmosphärischen Veränderungen und Turbulenzen standhalten.
    -Feuchtigkeit: Zu feuchte Umgebung beim Transport und bei der Lagerung der Produkte kann zu erheblichem Qualitätsverlust führen. Die richtige Verpackung schützt vor Feuchtigkeit.
    Produkteigenschaften: Wertigkeit, Zerbrechlichkeit, Produktgröße sowie Gewicht sind wichtige Faktoren bei der Auswahl der richtigen Packstoffe.
    -Temperatur: Die gewählte Verpackungslösung muss den Produkten eine optimale thermische Umgebung bieten. Dies gilt insbesondere bei verderblichen Produkten wie Lebensmittel.
    Es gibt viele Packmittel auf dem Markt. Dabei hat jedes Material seine Vor- und Nachteile.

-Begriff: Primär-, Sekundär- und Tertiärverpackungen %https://packpart.eu/glossar/packmittel/
    = hierarchische Reihenfolge der Produktverpackungen
    -Primär:
        -umhüllt das eigentliche Produkt und hat das Ziel, optimale Bedingungen für die Lagerung der Produkte zu schaffen
    -sekundär:
        -keinen direkten Kontakt mit der Ware und hält oft mehrere Primärverpackungen zusammen
        -Ziel: Schaffung der Kompaktheit und eines leichten Transports
    -tertiär:
        -vereint die sekundär verpackten Produkte und ermöglicht einen sicheren Transport in sehr großen Produktmengen (Paletten)


\section{Definition: Bauprodukt und Baustoff}
\label{sec:Definition: Bauprodukt und Baustoff}

    -Bauprodukt
            %https://eur-lex.europa.eu/legal-content/DE/TXT/?uri=CELEX:32011R0305
           [Der Ausdruck Bauprodukt bezeichnet] jedes Produkt oder jeden Bausatz, das beziehungsweise der hergestellt und in Verkehr gebracht wird, um dauerhaft in Bauwerke oder Teile davon eingebaut zu werden, und dessen Leistung sich auf die Leistung des Bauwerks im Hinblick auf die Grundanforderungen an Bauwerke auswirkt" aus VERORDNUNG (EU) Nr. 305/2011 DES EUROPÄISCHEN PARLAMENTS UND DES RATES vom 9. März 2011, KAPITEL I, Artikel 2, 1. Absatz (Bauproduktengesetz)
           
           %https://link.springer.com/content/pdf/10.1007/978-3-8348-9919-4.pdf
           Bauprodukte im Sinne des BauPG sind 
                1. Baustoffe, Bauteile und Anlagen, die hergestellt werden, um dauerhaft in bauliche Anlagen des Hoch- oder Tiefbaus eingebaut zu werden, 
                2. aus Baustoffen und Bauteilen vorgefertigte Anlagen, die hergestellt werden, um mit dem Erdboden verbunden zu werden, wie Fertighäuser, Fertiggaragen und Silos.
                
                Eine Besonderheit der Bauproduktenrichtlinie besteht darin, dass die wesentlichen Anforderungen nicht in Bezug auf das Bauprodukt selbst, sondern in Bezug auf das Bauwerk formuliert werden. Für die Anwendung der BPR sind deshalb weitere über den Richtlinientext hinausgehende Erläuterungen erforderlich; hierzu zählen die Grundlagendokumente und die Leitpapiere.
        
                    -Nach §3 Abs. 2 MBO dürfen Bauprodukte „nur verwendet werden, wenn bei ihrer Verwendung die baulichen Anlagen bei ordnungsgemäßer Instandhaltung während einer dem Zweck entsprechenden angemessenen Zeitdauer die Anforderungen dieses Gesetzes oder aufgrund dieses Gesetzes erfüllen und gebrauchstauglich sind.“
                    %https://link.springer.com/content/pdf/10.1007/978-3-8348-9919-4.pdf
            
            wesentlichen Anforderungen:(von Bauprodukten bzw eher Bauwerken, nicht von verpackungen)%-->diese müssen jedoch durch die verpackung erhalten bleiben!!!
                1. Mechanische Festigkeit und Standsicherheit
                2. Brandschutz
                3. Hygiene, Gesundheit und Umweltschutz
                4. Nutzungssicherheit
                5. Schallschutz
                6. Energieeinsparung und Wärmeschutz
            
            
        ( %http://www.derbrandschützer.de/2017/03/25/was-ist-eigentlich-ein-bauprodukt/
            Vereinfacht: Bauprodukte = Produkte, Baustoffe, Bauteile oder Bausätze
            
            Kriterien von Bauprodukten: 
                    -werden hergestellt oder als Anlage vorgefertigt
                    -werden dauerhaft in bauliche Anlagen eingebaut oder als vorgefertigte Anlage mit dem Erdboden verbunden
                        Beispiel:
                        -abnehmbaren Ausschmückungen oder bewegliche Ausscmückungen sind keine Bauprodukte (da nicht dauerhaft in bauliche Anlagen eingebaut
                        -festmontierter „Einbauschrank“ hingegen, könnte bei dieser Definition aber ggf ein Bauprodukt sein. laut alter vorordnung wäre der einbauschrank ein bauprodukt, laut neuer nicht
                    -wirken sich auf die Allgemeinen Anforderungen aus, die nach § 3    MBO 2016 an bauliche Anlagen gestellt werden
                        -Ein Bauprodukt muss sich zudem auch auf die allgemeinen Anforderungen, die nach § 3 MBO 2016 an bauliche Anlagen gestellt werden, „positiv“ auswirken:
                        -§ 3 Allgemeine Anforderungen:
                            "(1) Anlagen sind so anzuordnen, zu errichten, zu ändern und instand zu halten, dass die öffentliche Sicherheit und Ordnung, insbesondere Leben, Gesundheit und die natürlichen Lebensgrundlagen, nicht gefährdet werden; dabei sind die Grundanforderungen an Bauwerke gemäß Anhang I der Verordnung (EU) Nr. 305/2011 zu berücksichtigen.")
            
    -Baustoff:   ----> hierzu gibt es keine wirklichen definitionen
            wikipedia sagt: %https://de.wikipedia.org/wiki/Baustoff
                    -Ein Baustoff ist ein Material (in Form von Rohstoffen, Bauhilfsstoffen oder Halbzeug), das zum Errichten von Bauwerken und Gebäuden benutzt wird. 
                    - In der gesetzlichen Nomenklatur werden Baustoffe als Bauprodukt bezeichnet.
                    

\section{Definition Packmittel}
\label{sec:Definitionen:Definition Packmittel}

-!Begriff Packmittel!: (%https://packpart.eu/glossar/packmittel/)
    -Verpackung eines Produktes zu verstehen und bezieht sich auf Materialien, Gegenstände bzw. Vorrichtungen, die zur Lagerung, Handhabung, Transport, Präsentation sowie zum Schutz von Waren verwendet werden. Packmittel werden im gesamten Kreislauf integriert, angefangen vom Rohstoff bis zum Endprodukt, vom Hersteller bis zum Endverbraucher.
    -Arten von Packmitteln: Handels-, Sammel- und Transportverpackungen...(hierzu gibt es ganze kapitel)
    -Zweck von packmitteln:
        -Schutzfunktion
        -Prozessoptimierung
        -Optimierungsfunktion
        -Erleichterung der Buchhaltung
        -Nutzungsoptimierung
        -Informationsfunktion
    -Packmittelmaterial:
        -Metall
            -aus dünnem Metall:     -flexibel in der Handhabung
                                    -sehr widerstandsfähig
                                    -anpassungsfähig für verschiedene Zwecke
                                    -wird häufig als Rückverpackung verwendet(zb Faltständer und Stahlregale)
                                    -langlebig
                                    -guter Schutz der Produkte vor Erschütterungen während des Transports und grobem Umgang
                                    -Nachteil:  -Möglichkeit einer Korrosion
                                                -massives Gewicht
                                                
        -Kunststoff:    -universell
                        -nützlich für Einzelhandel und Transportwege
                        -geringen Gewicht
                        -Festigkeit
                        -langlebigkeit
                        -einfacher Reinigung
                        -guter Schutz vor mechanischen Schäden
                        -verhinderung von Eindringen von Staub
                        -feuchtigkeitsbeständig
                        -Nachteil:  -Entsorgen der Kunststoffverpackungen
                                    - damit verbundenen Auswirkungen auf die Umwelt
                                    
        -Wellpappe:     -geringen Kosten -->deshalb beliebt
        (=Karton?)      -leicht anpassbar an Form und Größe
                        -guter Schutz
                        -Nachteil:  -Keine große Widerstandsfähigkeit gegen
                                        -mechanische Einwirkung
                                        -Feuchtigkeit
        -Holz       -sicher
                    -gut für:   -große Produkte
                                -schwere Lasten
                                -Komplexe Lasten
                    -langfristige Lagerung gut möglich
                    -Nachteile:     -Sperrigkeit
                                    -geringe Flexibilität
        -Karton
        -Folie
         (=Kunststoff?)             
            
        Kriterien:  -abhängig vom Produkt(Gewicht, Größe, Beschaffenheit, Transportart
                    -Kosten(Logistikkosten,...)
                    -Lagerung
                    -Handhabung
                    -Transport
                    -Präsentation
                    -Schutz der Ware












