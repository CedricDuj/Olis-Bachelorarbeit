
%% Je nach Aufgabenstellung umfasst der Hauptteil unterschiedlich viele Abschnitte. Je nach Problemstellung umfasst der Hauptteil theoretische Grundlagen, erarbeitete Lösungsansätze mit Bewertung möglicher Lösungswege, ausgeführte Lösungen oder auch Funktionsnachweise mit vollständiger Beschreibung der der Test- und Untersuchungsbedingungen.


\chapter{Eignung von Verpackungen von Bauprodukten und Baustoffen}
\label{ch:Eignung von Verpackungen von Bauprodukten und Baustoffen}

\section{Identifizierung von Eignungskriterien für die Verwendung von Mehrwegverpackungen für Bauprodukte und Baustoffe}
\label{sec:Eignung von Verpackungen von Bauprodukten und Baustoffen:Identifizierung von Eignungskriterien für die Verwendung für Mehrwegverpackungen von Bauprodukte und Baustoffe}

%Die Eignungskriterien von Mehrwegverpackungen ergeben sich grundsätzlich dann, wenn eine Mehrwegverpackung im Vergleich zu einer Einwegverpqckung eine oder mehrere Vorteile bietet(finanziell, praktisch, ökologisch etc. ---> hier noch nachschauen). Dabei können die Gründe unterschiedlich sein.

%Die Eignungskriterien sind im Prinzip die Anforderungen?
                         
    Eignungskriterien: für die Verwendung von Mehrwegverpackungen für Bauprodukte und Baustoffe
            -überkapitel: Während der Lagerung, Während dem Transport(innerbbetrieblich und außerbetrieblich), evtl das Thema Laden(Einladen und Ausladen), danach entsorgung, kann aber sein dass es nicht mehr in meinen rahmen fällt
        -Ressourceneinsatz: 
                -Art der Ressource/Material
                -Menge der genutzen Ressource(für welche Menge Transportmaterial)
                        -->welche Meege ist notwendig um bestimmte artikel zu transportier
                -giftstoffe, schwermetalle, etc
                -ökologische Betrachtung und CO2 Ausstoß
        -Transportfähigkeit:
                -Widerstand gegen Beschädigung? -->Schutz
                -Gewicht
                -Handlichkeit und Kompatibilität mit Maschinen (zb. Stapler, ContainerLKW
                -Formstabilität bzw Verformbarkeit gewünscht oder nicht?
        -Weiterverarbeitung:(was ist damit gemeint?)(weiterverarbeitung statt entsorgung?
                -Lebenszyklus?
                -ökölogischer Abbau?
                -recyclingfähig
                -Umlauf (wie häufig etwas wiederverwendet werden kann)
                (%(https://packpart.eu/glossar/mehrwegverpackung/)
        -Kosteneinsparung
                -langen Lebenszyklus/Wiederverwendbarkeit (+Entsorgung)
                -geringe anschaffungskosten
                -evtl höhere anschaffungskosten aber sparen bei langer Nutzung
        -Lebenszyklus und Umsatz
        -Resistenz gegen Umwelteinflüsse:
            -UV-Strahlung -->Porösigkeit(von zb Plastik)
            -Regen(+feuchtigkeit! allgemein)-, Schnee- (und Anprall), Brand-, Rauch- , Wärme!-, und Schallschutz
            -robust, kratz(+abrieb)fest und unempfindlich gegenüber Verschmutzungen, enorm belastbar
        -laufende Kosten
            -Sammelstelle einrichten
            -Reinigung und Wiederaufwertung
            -Pflegeaufwand
            -etablierte Systeme einrichten die überhaupt "Mehrweg"verpackungen annehmen und bei verschiedenen Firmen(bei bereits ausgebreiteteten Systemen über meherere Firmen) -->Logistik!
        -nachhatigkeit(mit zertifizierung(zb bei holz))(Wechselwirkung mit Weiterverarbeitung)(generelles ziel?!)
            -regenrative rohstoffe
        (-transparenz(durchschauen-überblick)(eher psychologischer faktor))
        
%https://www.medewo.com/blog/de/loesungen/was-muss-eine-verpackung-leisten/
    -Schutz der Ware von A nach B
    -über den Inhalt und mögliche Sicherheitsmaßnahmen informieren
    -beim Transport und Laden sichere sowie effiziente Abläufe ermöglichen
    -das Produkt auf ansprechende Weise präsentieren
    -->Sinn aller Faktoren: Werterhalt der Ware
            -->im weiteren Sinne: Den Wert erhalten bedeutet aber auch, dass diese mit dem Produkt und Ihrem (Versand-)Service zufrieden sind.
            -->Das aussehen der Verpackung vermittelt auch den Wert der Ware(Design sowie eventuelle Beschädigung)
    
    Faktoren:
        -mechanischer Beanspruchung(Stöße, Druck, Erschütterung), klimatischen Einwirkungen(Feuchtigkeit, Kälte, Wärme) sowie Handling durch den Menschen(unsachgemäßes Laden, Transportieren oder Einlagern)
        -entgegen wirken: 
            -Stabilität, Konservierung, Markierung \& Schutzmaßnahmen
                -Stabilität: 
                    -passenden Umkarton wählen, je nach Größe und Gewicht der Ware
                    -Hohlräume vermeiden durch Polster- und Füllmaterial, verhindert Rutschen und garantiert zusätzlichen Schutz der Ware
                    -den richtigen Klebeband-Verschluss wählen. bei schwerer Ware andere Verschlussarten wählen. zb doppel-t-verschluss
                    -auf ausreichende Ladungssicherheit achten(auf der Palette)
                        -->durch Umreifungsbänder und Strech-Folie, bei unförmiger Ware-->Schrumpfen, da sich die Folie dann an die Ware anpasst
                -Konservierung:
                    -Witterungsbedingungen beachten
                        -Schmutz und Feuchtigkeit, Nässe
                    -bei Temperatursensibler Ware auf Thermoverpackung zurückgreifen
                        -Luftpolsterfolie, Thermo-Palettenhaube oder Box aus EPS
                            -isoliert vor Kälte oder Wärme
                        -Korrosionssschutz bei Metallprodukten wichtig
                            -Verpackung mit Trockenmitteln oder VCI
                        -Wahl des Klebebandes: bei Transport über verschiedene Zeitzonen überzeugt ein Klebeband mit Naturkautschuk aufgrund hoher Temperaturbeständigkeit
                -Markierung und Schutzmaßnahmen
                    -Markierung des Packgutes mit Sicherheitshinweisen zur entsprechenden Handhabung
                        -bsp: vorsicht zerbrechlich, kühl lagern, Gefahrgut-Klassen
                    -Sicherheitsklebebänder bei teurer Ware, schützt vor unbemerktem öffnen und Manipulation/Betrug
                    - bei Produkten, die bei geringem Neigungswinkel Schaden nehmen können, kann ein Kippindikator helfen
        -->dieses Gleichgewicht entscheiden über den Werterhalt und die Unversehrheit der ware  

--------------------------------------
        %https://packpart.eu/glossar/packmittel/
        Kriterien:  -abhängig vom Produkt(Gewicht, Größe, Beschaffenheit, Transportart
                    -Kosten(Logistikkosten,...)
                    -Lagerung
                    -Handhabung
                    -Transport
                    -Präsentation
                    -Schutz der Ware
             --> das sind kriterien, die wichtig bei Packmitteln sind       
--------------------------------------


-Kriterien für die Wahl eines Verpackungsmaterials:(v.a. für den optimalen Schutz) %(https://packpart.eu/glossar/packmittel/)
    -Transport: Bei der Wahl des Packmittels ist sicherzustellen, dass die Produkte auf dem Transportweg dank des Packmittels gut geschützt werden. Zum Beispiel muss per Luftfracht transportierte Ware atmosphärischen Veränderungen und Turbulenzen standhalten -->eher bei Spezialfracht. nicht bei normaler Anwendung im Bauwesen. 
    -Feuchtigkeit: Zu feuchte Umgebung beim Transport und bei der Lagerung der Produkte kann zu erheblichem Qualitätsverlust führen. Die richtige Verpackung schützt vor Feuchtigkeit.
    Produkteigenschaften: Wertigkeit, Zerbrechlichkeit, Produktgröße sowie Gewicht sind wichtige Faktoren bei der Auswahl der richtigen Packstoffe.
    -Temperatur: Die gewählte Verpackungslösung muss den Produkten eine optimale thermische Umgebung bieten. Dies gilt insbesondere bei verderblichen Produkten wie Lebensmittel.
    Es gibt viele Packmittel auf dem Markt. Dabei hat jedes Material seine Vor- und Nachteile.
    
%TODO: Löschen
\begin{figure}
    \centering
    \includegraphics[scale=0.1]{./images/Penner}
    \caption{Caption}
    \label{fig:my_label}
\end{figure}

-------------------------------------------------------------------------------------------

\section{Eignung von Mehrwegverpackungen für typische Bauprodukte und Baustoffe}
\label{sec:Eignung von Verpackungen von Bauprodukten und Baustoffen:Eignung von Mehrwegverpackungen für typische Bauprodukte und Baustoffe}

%Die Eignung von Mehrwegverpackungen hängt stark von den Eignungskriterien ab. also die erst machen bzw sich überlegen welche es sein könnten.

Typische Bauprodukte und Baustoffe: (Definition)
%-->darauf referenziert die Eignung von MEHRWEGverpackungen
-abhängig von den Anforderungen
typische Bauprodukte und Bautoffe (unabhängig von der Verpackung) wählen:
    
    -Aufbau nach Baubetrieb:
    -Rohbau:
        -Aushub(Erde)
        -Beton
            -Betonabstandshalter
            -Fertigteile aller Art
            -Bei Ortbeton:
                -Wasser
                -Zement
                -Steine/Kies(Grob- und Feinkorn)
                -Sand
                -zusatzstoffe und -mittel
        -Stahl
            -Bewehrung
        -Rohre(Entwässerung
        -Dämmung
            -Dämmwolle (Mineralfaserwolle, etc)
            -Styrodur(synthetische Dämmung?)
        -Folie(zum schutz von Wassereintritt(im keller))
        -Bitumen (für den Keller)
        -Noppenfolie
        -Mauerwerk
            -Kalksandstein
            -Ziegel
            -Mörtel
            -Zement
            -Yttong(richtiger Name pls)
        -Holz
            -Sparren und Pfetten
        -Ziegel fürs Dach
        
        
        Brandschutzdecken und Gipsfaserplatten, Rahmenprofile und Ausgleichsschüttung
        -Dämmstoffe:
            -->Schall- oder Wärmedämmung, technische Isolierung von heizungen, fachgerechte Installation des Brandschutzes
        -Bauelemente:
            -->Fenster und Türen, Tore und Treppen
        -                    -Fassade
                        -->Vom Wärmedämm-Verbundsystem über Verblendmauerwerk und Außenputz bis hin zur Metall- oder Holzfassade
                        --> für private Wohnhäuser sowie für Gewerbe- und Industriegebäude
                    -Holz:
                        -->Unterbau: Für die Konstruktion von Dachstühlen und Wänden erhalten Sie bei uns Plattenware, Schnittholz und Co
        -Fliesen
            -verschiedene Materialien: zb Keramik, Naturstein(Steinzeug), Holz, Laminat, PVC, marmor, granit, quarzit, sandstein, terracotta
                -->Unterscheidung beim Transport und der Verpackung
        -Garten und Landschaftsbau:
            -betonplatten/betonfliesen vs natursteinfliesen
            -Zäune
                -Holzzäune, Maschendrahtzaun, WPC-Zaun(Wood Plastic Composite) =Verbundwerkstoff, Metallzaun
            -Pflastersteine
                -aus: Beton, Naturstein(zb Basalt), Hochofenschlacke, Beton mit Natursteinzusätzen
            -Terassendielen
        -Zierkies und -split
        -Holz
            -->Kantholz, Rundholz, Bauschnittholz, Holzplatten oder Holzbalken           -->Konstruktionsvollholz (KVH), Schnittholz, Schalungen und Bretter sowie Brettschichtholz (BSH) und Hobelware
            -->Holzkonstruktionen (Dachstühle; Nagelbinder,Holzrahmenbau), Holzwerkstoffe, Holzfaserdämmstoffe, Terrassenholz und Balkonbeläge
            -->Konstruktionsvollholz (KVH), Brettschichtholz (BSH), Hobelware, Sperrholz, Terrassen-Beläge, OSB-Platten, Spanplatten,  Holzdämmstoffe, sowie ein umfangreiches Sortiment an sägefrischem Schnittholz
        -Parkett, Laminat und Vinyl-Boden-->stukateur
        -Putz
        -Türen, Tore \& Fenster
        -Elektrikersachen
        -Heizungsbauer
            -Rohre und Heizung
        

    

gewisse Ordnung erstellen bzw in grobe Verpackungssysteme runterbrechen
    -->diese dann mit den Eignungskriterien koppeln und evtl sogar eine Art "Tabelle erstellen mit den Eignungskriterien auf der Abszisse und typischen Bauprodukten/Bauproduktverpackungen auf der Ordinate.
        
    
Die Eignung von Mehrwegverpackungen für typische Bauprodukte und Baustoffe ist abhängig von den Anforderungen an die Ware, die durch Verpackungen gewährleistet werden muss.

Als genutzt Mehrwegbehälter werden hier folgende ... genannt :

        Europaletten, sowie Pfandpaletten:
        %https://www.hausjournal.net/europalette-belastung :
            -Punktbelastung: 1000kg/1t
            -Flächenbeslastung: 1500kg / 1,5t; EPAL-Palette bs 2t(quelle eins drunter)
        -%https://www.vallee-partner.de/blog/paletten :
            -gefertigt aus Holz, Restholz, Kunststoffgranulat oder Recyclingkunststoffen (90 Prozent aus Holz),Plastikpaletten, Paletten aus Wellpappe, sowie Metall- und Papierpaletten
            -etwa 20 Prozent der gesamten Holzproduktion in Europa auf das Konto von Holzpaletten und -verpackungen
            -Holzpaletten werden vorwiegend aus Produktionsresten wie der Seitenware von Nadelhölzern gefertigt
            -Eine Europallette gilt als nicht mehr tauschbar und muss repariert werden, wenn beispielsweise Absplitterungen oder Brüche die sichere Nutzung gefährden oder Bretter, Klötze oder Markierungen fehlen.
            -Nicht mehr reparierbare und daher unbrauchbare Paletten, die die Reparaturprüfung nicht überstehen, werden meist in ihre brauchbaren Einzelteile zerlegt, welche wieder zu Paletten verarbeitet werden können. Die zerstörten oder morschen Teile werden im Schredder oder einer Entsorgungsanlage vernichtet; dies geschieht in eingetragenen, von der EPAL lizensierten Betrieben
            -Upcycling von EPAL Paletten ist noch ein thema auf das genauer eingegangen wird in der quelle.
            -für die lagerung und des transport, sowie be- und entladen
            -als untergrund für bigpacks
            -für Dachziegel
            -für betonprodukte
                -rohre,fliesen,L-shape, 
            -für eimer
            -für plastikkanalrohrsseme
            -zementsäcke
            -metalleinzelteile
                -stäbe
            -dämmung(aufgerollt)
            -Steine aller Art, die annähernd rechteckig sind
                -kalksandstein
                -yttong
            -als "Unterboden" für Gitterboxen und deren Transportware

        -gitterboxen/euroboxen /Boxpaletten
            -für rohre, loses material,für kleinere Steine
            -mit und ohne deckel erhältlich
            -für Kran oder nur Stapler geeignet
            -Sichert die Ware (auf einer Palette)
            -%https://de.wikipedia.org/wiki/Gitterbox
            -bietet die Möglichkeit des Stapelns von unförmiger Ware
            -Traglast (1000 bis 1500 Kilogramm),
            -die Auflast (4400 bis 6000 Kilogramm)
            -3-5 Stück stapelbar
            -->Möglichkeit einer Blocklagerung besteht
                -->bis zu sieben gestapelten Gitterboxen üblich(in der Blocklagerung)
            -Eigengewicht ca 70 kg
            -Anwendung bei instabilen und druckempfindlichen Güter (bei der Lagerung)
            -Die vorhandenen zwei Vorderwandklappen sind drehbar um eine waagerechte Achse gelagert. Die Klappen dienen der besseren Entnahme der geladenen Ware.
            -Es gibt auch faltbare Gitterboxen, die beim Rücktransport bis zu 80 Prozent Platz sparen
            -"Halbhohe Gitterbox" ist etwa halb so hoch, bei sonst gleichen Abmessungen
            -"Hohe Gitterbox" existiert(Höhe 1,8m)
                - Nutzvolumen von etwa 1,6 Kubikmeter und vorne eine Klappe
            -"Große Gitterbox"
                -bei doppelter Breite, bei sonst gleichen Abmessungen wie die hohe Gitterbox, ein Nutzvolumen von etwa 3,2 Kubikmetern. Auch sie hat eine Klappe vorne.
        Siehe auch:
            -DIN EN 13626, Verpackung – Boxpaletten – Allgemeine Anforderungen und Prüfverfahren:
            DIN 13626:2003:
                -Vorgesehen für die wiederholte Anwendung
                -behalten ihre gebrauchstauglichkeit und sichere handhabung
                -werden verwendet für: 
                    -mechanische Handhabung mit gabelstapler und Handhubwagen
                    -Massenlagerungen in stapellagern, bei denen es aus Sicherheitsgründen nicht ratsam ist, Boxpaletten bis zu einer Höhe zu stapeln, die das Siebenfache der kleineren horizontalen Abmessung der Boxpaletten überschreitet
                    -Transport
                -Diese Europäische Norm gilt für Boxpaletten, Rungenpaletten und Gitterboxpaletten, nicht aber für Silo- und
                -Sie können von fester Bauart, zusammenlegbar oder zerlegbar sein. Die Produkte, die in dieser Norm behandelt werden, sind Boxpaletten, die mit Gabelstaplern oder Hubwagen, nicht jedoch mit anderen Hebevorrichtungen gehandhabt werden können Tankpaletten
                -Boxpalette, Rungenpalette oder Gitterboxpalette
                    -entweder von fester Bauart, zusammenlegbar oder zerlegbar
                -verschiedene Lastrechnungen:
                    -Nennlast, Nennstapellast, dynamische Last, Prüflast
                -Anforderungen eine Gitterbox:
                    -muss mit Stapelvorrichtung versehen werden
                    -Handhabung vom Boden mit Gabelstapler/Hubwagen ermöglichen
                    -die Höhe der Boxpalette darf nicht das doppelte des kleinsten Bodenmaßes überschreiten
                -Prüfungen
                    -Prüflast: 2 Verfahren
                    -Klimatische Konditionierung:
                        -keine erforderlich für Boxen, wenn vollständig aus Metall
                -es gibt Kunststoffpaletten, Holzpaletten, Holzfaserpaletten
            -Europaletten, Container, Fahrzeug(Sattelauflieger) sind Transporthilfsmittel und damit Teil der Ladungsträger!
            
        -big bag/Bulk bag/ Jumbo bag/ FIBC(=Flexible Intermediate Bulk Container)
            (-überwiegend schüttmterial- kies, sand, ziersplit.)
            -Anwendungsbereich
                -feste Füllgüter in Pulver,- Granulat,- Pastenform
            -Arten von FIBC(BigBags)
                -flexible Großpackmittel(FIBC)
                    -bestehen aus flexiblem Material wie Gewebe, Kunststofffolie, Papier und sind so konstruiert, dass sie mit der Füllung in in Berührung kommen, entweder direkt oder durch einen Inliner, und im leeren Zustand zusammenfaltbar sind
                -Mehrweg-FIBC für hohe Beanspruchung
                    -ein FIBC, vorgesehen für eine Vielzahl von Befüllungen und Entleerungen, der sowohl beim Hersteller und/oder Anwender in einer Weise repariert werden kann, dass die Zugfestigkeit nach Reparatur mindestens so groß ist wie im Original
                -Mehrweg-FIBC für normale Beanspruchung
                    -ein FIBC, vorgesehen für eine begrenzte Anzahl von Befüllungen und Entleerungen
                    -arf im Falle einer Beschädigung nicht wiederverwendet werden, d. h., er ist nicht reparierbar
                    -Das Auswechseln eines herausnehmbaren Inliners wird nicht als Reparatur angesehen
                -Einweg-FIBC
                    -ein FIBC, vorgesehen für nur eine einmalige Befüllung
                -Nenntragfähigkeit (SWL)
                    -die maximale Last, die der FIBC laut Zertifikat im Einsatz tragen darf
                -FIBC.Teile/Komponenten
                    -Mantel
                        -ein ein- oder mehrlagiger Schlauch, der nahtlos oder aus einem oder mehreren Teilen zusammengesetzt ist
                    -Boden
                        -jener Teil des FIBC, der mit dem Mantel verbunden ist oder eine Einheit mit ihm bildet und die Grundfläche des stehenden FIBC bildet
                    -flacher Boden
                        -Boden ohne Öffnung
                    -Boden mit Öffnung
                        -flacher, konischer oder anders geformter Boden mit einer Öffnung
                    -ganz offener Boden
                        -Verlängerung des Mantelgewebes, das nach dem Verschließen den Boden des FIBC bildet
                    -Deckel
                        -der obere Teil des FIBC, ohne Hebevorrichtungen, der nach dem Verschließen den Deckel des FIBC bildet
                    -Körper
                        -der Mantel und der Boden des FIBC
                    -Inliner
                        -integrierter oder herausnehmbarer Behälter, der in den FIBC passt
                -Betriebsvorrichtungen
                    -Einfüllöffnung, Einfüllstutzen, Einfüllschlitz, Auslauf, Auslaufstutzen, Verschlüsse(Gurte, Seile, Riemen usw., die zum Verschließen der Einfüll- und Auslauföffnungen verwendet werden)
                -Handhabungsvorrichtungen
                    -Stütz und Hebevorrichtungen
                        -Gurte, Schlaufen, Seile, Ösen, Rahmen oder andere Vorrichtungen, die aus Verlängerung des Mantels des FIBC gebildet werden oder integriert oder abnehmbar am FIBC angebracht sind und für das Stützen oder das Heben des FIBC verwendet werden
                    -Aufhängung(Vierpunkt, Dreipunkt, Einpunkt)
            -Werkstoff:
                Die Eigenschaften der Materialien dürfen durch geeignete Zusätze modifiziert werden, um die Beständigkeit des Materials, z. B. gegen Schwächung durch Wärme und Sonnenlicht, zu verbessern und die Wirkung von statischer Elektrizität zu verringern.
            
        -container

-Ladeeinheit (LE)
    besteht aus: 
        -Ladehilfsmittel: z. B. Palette, Container, Tablar, Gitterboxpalette, Unit Load Device
        -Ladeeinheitensicherungsmittel
        -Packstück
    -Unterteilung in:
        -LE mit tragender Funktion: (Ladungsträger)
            -v.a. Paletten aus Holz, Kunststoff und Metall, aber auch:
                -Industriepalette, Rungenpalette, Rollpaletten, Einwegpaletten
            -Stapelung ist vierfach übereinander möglich
        -LE mit umschließender Form:
            -v.a. Gitterboxpaletten
                -bestehen aus:
                    -drei festen Gitterwänden, einer abnehmbaren Vorderwand,einer Bodenfläche (siehe Gitterboxen)
            -neben Gitterboxen gibt es noch:
                -Vollwandboxpaletten,
                -spezielle Klappboxen und Faltboxen
                    -->bieten beim Rücktransport eine Volumeneinsparung von 60 %
        -LE mit abschließender Form:
            -v.a. Container, aber auch  Wechselbrücken
    
       
------------------------------------------------------------------------------------------

\section{Potentiale und Herausforderungen verschiedener Mehrwegverpackungen}
\label{sec:Eignung von Verpackungen von Bauprodukten und Baustoffen:Potentiale und Herausforderungen verschiedener Mehrwegverpackungen}

%dieses kapitel gehört eigentlich zur Eignung von Mehrwegverpackungen. Hierbei sollen die potentiale identifiziert werden. Operator: Identifizieren.

hier eigentlich die verschiedenen Verpackungssysteme einfach hinschreiben die ich im Raab kercher gesehen habe(die mehrwegverpackungen eben)
    -->Einwegverpackungssysteme dann in das nächste Kapitel?

------------------------------------------------------------------------------------------
