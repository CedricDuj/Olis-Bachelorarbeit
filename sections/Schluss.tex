


%% Dieser Teil der Arbeit soll die Arbeit abschließen. Ziel ist, die erarbeiteten Ergebnisse für den Leser zusammenzufassen und diese letztendlich kritisch zu betrachten und eventuell mögliche weitere Schritte kurz aufzuzeigen.





\chapter{Schluss}
\label{ch:Schluss}

-fasst die wichtigsten Ergebnisse prägnant zusammen und stellt somit den Höhepunkt deiner Bachelorarbeit dar

-steht im direkten Zusammenhang mit der Einleitung, da du auf die Forschungsfragen oder Hypothesen eingehst, die zu Beginn der Bachelorarbeit aufgestellt wurden

-Die Länge des Fazits ist abhängig vom Umfang deiner Arbeit, sollte jedoch ca. 5–10 Prozent der gesamten Bachelorarbeit ausmachen

-Keine neuen Informationen und Interpretationen
-Keine Beispiele und Zitate:
    -Im Fazit fasst du Fakten zusammen und erklärst sie nicht anhand neuer Beispiele und Zitate anderer Forschenden
-Dein Ergebnis ist immer wertvoll
    -Es kann vorkommen, dass deine Ergebnisse nicht deinen Erwartungen entsprechen. Wenn du deine Forschungsfrage aber gut gestellt hast, bspw. mit den Formulierungen ‚wie viel‘ oder ‚inwiefern‘, wirst du immer ein wertvolles Ergebnis erhalten, das die Forschung auf diesem Gebiet weiterbringt.
-Beim Präsentieren der Fakten verwendest du in einem Fazit das Präsens. Wenn du auf deine Forschung verweist, benutzt du das Präteritum.

-Elemente eines Fazits:
    -Abschließende Beantwortung deiner Forschungsfrage.
    -Gesamtdarstellung deiner wichtigsten Ergebnisse.
    -Zusammenfassung der Schlussfolgerungen aus dem Diskussionsteil (z. B. bezüglich offen gebliebener Fragen oder Ausblicke auf zukünftige Forschung).
-keine Elemente eines Fazits:
    -Wiederhole nicht nur Formulierungen aus deinem Hauptteil.
    -Sei nicht zu bescheiden, sondern zeige, was du erreicht hast.
    -Bringe keinen neuen Ideen, Studien oder Beispiele ein.
    -Schreibe das Fazit nicht kurz vor knapp, sondern nimm dir genügend Zeit.
    
-Unterschied zwischen Fazit und Diskussion

Oft kann der Unterschied zwischen dem Fazit und der Diskussion ein wenig verwirrend sein. Diese Tabelle schafft Klarheit und erklärt die wichtigsten Unterschiede:

Fazit:       
    -Zusammenfassung 	
    -Kurz und bündig 	
    -Gesamtdarstellung 	
    -Keine Beispiele und neuen Informationen 	

Diskussion:
    -Interpretation
    -Ausführliche Behandlung der Resultate
    -Evaluation
    -Beispiele und neue Informationen

-------------------------------------------------------------------------------------------

\section{Fazit/Zusammenfassung der Kernthese}
\label{sec:Schluss:Fazit/Zusammenfassung der Kernthese}

\section{Schlussfolgerung}
\label{sec:Schlussfolgerung}

\section{kritische Würdigung}
\label{sec:kritische Würdigung}

\section{Ausblick}
\label{Ausblick}


auch evtl einen Ausblick auf eine weitere anschließende bachelorarbeit
        - wo wrde diese anschließen
        -welche fragen/problemzonen könnte man darin behandeln?