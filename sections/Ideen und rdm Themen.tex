

-Lebenszyklus und Umlauf als ein extra Thema machen?
-Es fehlt der rote faden/die Struktur, ich weiß nicht wie ich weiter machen soll
-ich weiß nicht inwiefern Themen zu meiner Bachelorarbeit gehören zb:
    -Entsorgung - Lz-mw
    -Recycling -LZ-mw
    -

-momentane idee für eine alternative zu einwegverpckunge ist eine langlebige umweltfreunliche varinate aus entweder plastik mit holz gemischt oder reinem plastik oder reinme holz, die vielfältig geformt und gebaut werden kann und durch hohe stückzahlen preislich rentabel ist. wichtig hierbei ist die robustigkeit und die langlebigkeit, die einfache säuberung und ein umweltfreundlihe entsorgung oder recycling am ende des lebenszyklus

------------------------------------------------------------------------------------------
Ab hier rdm
------------------------------------------------------------------------------------------

-Begriff Entsorgung: sowohl im Bezug aus einweg als auch mehrweg-->Indikator
--> Mehrweg: 


%(https://www.bmuv.de/faqs/mehrwegverpackungen/)
-Wieso gibt es dennoch Einwegverpackungen?
    -Verbot von bestimmten Verpackungen oder Verpackungsarten ist mit den europarechtlichen Vorgaben der Verpackungsrichtlinie nicht vereinbar.
    -Die Einwegkunststoffrichtlinie ermöglicht nur Maßnahmen zur Reduzierung des Verbrauchs von Einwegbechern, kein allgemeines Getränkebecherverbot. 
    -Komplettverbot zurzeit auch ökologisch nicht sinnvoll
        -->Bsp: Für Reisende und Pendler ist es unmöglich, einen am Reiseort erworbenen Becher zurückzugeben. Die würden wohl eher im Müll landen

-psychologische Faktoren der Käufer, einen Mehrwegbecher zu fordern.
    -->Hemmschwelle

-pfand system auch sinnvoll für das Bauwesen?
%(https://www.wlw.de/de/inside-business/branchen-insights/verpackung/einweg-und-mehrwegverpackungen)


-Vor und Nachteile von Mehrwegtransportverpackungen


------------------------------------------------------------------------------------------


roter faden: wieso brauche ich überhaupt mehrwegverpackungen? was ist an ihnen besser als an einwegverpackungen?

    -was ist eine Einwegverpackung?
    -aktueller Praxisstand zum einsatz von einwegverpackungen
    -vor und Nachteile von Einwegpackungen?
    
    -was ist eine mehrwegverpackung?
    -aktueller praxisstand von mehrwegverpackungen
    -vor und nachteile von mehrwegverpackungen?
    
    (-weshalb gibt es nur so wenig mehrwegverpackungen?)
    
    -Auswirkungen und Möglichkeiten im Rahmen des Verpackungsgesetzes in Bezug auf nachhaltige Veränderungen im Bauwesen + grüner punkt
    
    (-differenzierung von subbaubranchen und unterschiede in der Anwendung von verpackungen)
    
    
    -identifizierung von passenden eignungskriterien für die verwendung von mehrwegverpackungen 
        (wie z.B. Ressourceneinsatz, Transportfähigkeit, Weiterverarbeitung etc.)
        
          -nutzbarkeit von anderen am markt bereits verfügbaren systeme für das bauwesen
    
    -hauptteil: 
        
        -aufbauend auf den kriterien die eignung von mehrwegverpackungen für typische bauprodukte
            -hier evtl eine abstufung welche arten von bauprodukten es gibt und wie sie sich unterscheiden
                -->inwiefern ändern sich die verschiedenen Arten für die Eignung? 
        
        
------------------------------------------------------------------------------------------        

Neues aus der Besprechung:
    1.1 wissenschaftliche Motivation stat der persönlichen Motivation - check
    
    für das grundlagen kapitel und den hauptteil  andere namen wählen - check
    
    wenn ich unterkapitel wähle dann immer 2 wählen *(da eine Unterteilung/unterscheidung)
    
    falls ich etwas zu biologisch abbaubauren verpackungen sagen möchte dann bei geringem umfang in den ausblick und bei größerem unfang bietet sch es an ein eigenes kapitel im hauptteil zu machen
    
    meine bachelorarbeit bezieht sich jedoch auf mehrwegverpackungen! nicht auf alternative einwegverpackungen
    
    Methodik ist ein eigenes kapitel!
    nicht nur ein absatz sondern ein wirkliches kapitel
    alles was ich zum gelingern der arbeit brauche und das vorgehen
    
    experteninterview kann sehr hilfeich sein. vor allem zu mehrwegverpackungen
    und schwierigkeiten und potenziale 
    1-3 experteninterview
    möglich per video
    oder gedankenprotokoll --> das dann in den anhang
    
---------------------------------------------------------------------------------
    
    ideen von heute Abend:(30.06.22)
        
        - beim bauhof verschieden produkte suchen und daraus schließen ob es eine art standardbehälter verschiedener produkte gibt. wenn ich diese finde, muss ich schauen, ob se aus einwegbehältern bestehen und wenn nicht wieso nicht.
            -->daraus generell mal informieren, welche art von behälter und verpackungen gemeint sind
        -schauen bei raab gercher und union(Baucenter)
        -was wird überhaupt alles eingepackt im Bauwesen?
            --> nicht nur material wie stahl oder betonabstandshalter, sondern auch sowas wie handschuhe dei ab und zu gebraucht werden, alles was die arbeiter zum leben brauchen wie toilette, getränke etc. auch drumherum mal schauen.
            
            -->abhängig von verschiedenen Gewerken!
        -Rohbau, Ausbau, Dach, Holz, Fliesen, Bauelemente, Tiefbau-Versorgung und -Entsorgung, Natursteine, Garten- und Landschaftsbau, Baugeräte und Werkzeuge sowie Produkte für den Arbeitsschutz (%https://www.raabkarcher.de/sortiment/)
                                -->gibt es unterschiede zwischen Bauunternehmer oder Ein-Mann-Betrieben, Großbaustelle oder kleineren Bauvorhaben? wenn ja welche?
                            
-----------------------------------------------------------------------------------------

    Raab Karcher Sortiment:
                            
                -Arbeitskleidung und Arbeitsschutz:
                    %https://www.raabkarcher.de/sortiment/arbeitskleidung-arbeitsschutz/
                    -Arbeitskleidung bei Sonne, Wind, Regen, Schnee, hohen und tiefen temperaturen, flammen(meine meinung: meist nicht aus einwegmaterialien, da handwerker viel draußen sind)
                    -komfort, schutz, stauraum für werkzeuge
                    -"Arbeitshosen und Shorts sind strapazierfähig, abriebfest und sitzen immer bequem"
                    
                      -handschuhe, brillen, helme, sicherheitssschuhe, feinstaubmaske, gehörschutz, gummistiefel, schutzanzug, warnweste, sonnenschutz(kappen)
                        -->acessoires und kleinere Dinge, dei nicht zu Baustoffen oder Materialien zähen aber dennoch gebraucht und verwendet werden, vor allem kleinere, für menschen gemachte utendilien, die ebenfalls bestellt oder gekauft werden müssen.
                        
                -Dämmstoffe und Trockenbausysteme: %https://www.raabkarcher.de/sortiment/ausbau/
                    -Wand, Boden, Decke
                        -->Brandschutzdecken und Gipsfaserplatten, Rahmenprofile und Ausgleichsschüttung
                    -Dämmstoffe:
                        -->Schall- oder Wärmedämmung, technische Isolierung von heizungen, fachgerechte Installation des Brandschutzes
                    -Bauelemente:
                        -->Fenster und Türen, Tore und Treppen
                -Baugeräte & Werkzeuge:
                %https://www.raabkarcher.de/sortiment/baugeraete-werkzeuge/
                    -meines erachtens alles nur mehrwegartikel, nicht wird nur einmal verwendet
                -Dach:
                %https://www.raabkarcher.de/sortiment/dach/
                    -Fassade
                        -->Vom Wärmedämm-Verbundsystem über Verblendmauerwerk und Außenputz bis hin zur Metall- oder Holzfassade
                        --> für private Wohnhäuser sowie für Gewerbe- und Industriegebäude
                    -Holz:
                        -->Unterbau: Für die Konstruktion von Dachstühlen und Wänden erhalten Sie bei uns Plattenware, Schnittholz und Co
                -Fliese:
                %https://www.raabkarcher.de/sortiment/fliese/
                    -Unterschiede von fliesen, die sich auch im transport und der verpackung unterscheiden:
                        -Fließen im Bad
                        -großformatige fliesen/kleinformatige fliesen
                        -verschiedene Materialien: zb Keramik, Naturstein(Steinzeug), Holz, Laminat, PVC, marmor, granit, quarzit, sandstein, terracotta
                        -abriebfestigkeit(abriebklassen), glasiert, unglasiert
                -Garten und Landschaftsbau
                %https://www.raabkarcher.de/sortiment/gartenlandschaftsbau/
                        -von Terrassenpflaster und Betonplatten über Keramikplatten bis hin zu Naturstein und WPC-Terrassendielen
                    -betonplatten/betonfliesen vs natursteinfliesen(auch außen)
                    -Fugenmörtel, Bettungsmörtel und Haftschlämme
                        -->Fugenmörtel aus einem Eimer zu kaufen
                        -->Zementfugenmörtel, ontaktschlämme und Bettungsmörtel aus einer "Tüte"
                    -Gartenzäune:
                        -Holzzäune, Maschendrahtzaun, WPC-Zaun(Wood Plastic Composite) =Verbundwerkstoff, Metallzaun
                    -Terassenfliesen und Balkonfliesen
                        -->überwiegende Überschneidung mit dem Thema fliesen und Betonplatten im ausenbereich
                    -Palisaden:
                        --> = kleine Pfähle, die bei der Gestaltung der Terrasse und des Gartens zum Einsatz kommen
                        -->Palisaden gibt es in ganz unterschiedlichen Höhen und Stärken
                            --Y 20cm - 1,5m
                        -->aus verschiednen Materialien:
                            -->Rundhölzer(spezielle Behandlung: Kesseldruckimprägnierung), Naturstein, beton -->für den Einsatz im Freien wappnen
                        -->breites Spektrum an Formen, Oberflächen, Höhen und Farben
                            -->rund, quadratisch, rechteckig
                        -->Wichtig beim Setzen ist ein geeignetes Fundament, das zum         Beispiel aus Magerbeton
                    -Pflastersteine:
                        --> lassen sich mit ihnen kleine Pfade, der Hof oder eine Terrasse pflastern
                        -->Steine stehen in verschiedenen Farben, Größen und Formen zur Verfügung
                        --> bestehen aus Beton, Naturstein(zb Basalt), Hochofenschlacke, Beton mit Natursteinzusätzen, 
                        -->Unzählige Farben, Formen und Oberflächen
                        -->verschleißbeständig und robust
                        -->Nachhaltigkeit und Umweltfreundlich
                            -->Die Herstellungsverfahren belasten die Umwelt nicht
                            -->sie bemühen keine Ressourcen
                            -->das Material ist wiederverwertbar
                        -->Bei der Verlegung ist ein tragfähiger Unterbau das A und O
                            -->mit Trägerschicht
                    -Stele:
                        -->aus Granit, Holz, Schiefer, Basalt, Glas, Keramik, Stahl und Beton
                        -->am Boden verschraubt oder einbetoniert 
                            -->40 bis 50 Zentimeter
                    -Terrassendielen aus Holz:
                        -->witterungsbeständig, Lasuren und Öle, 
                            -->gut gegen tierische und pflanzliche Schädlinge 
                        -->Unterkonstruktion notwendig
                            -->für Stabilität und gegen Materialspannung
                        -->Besonders lange Dielen vs kurze dielen
                        -->Pflegeöl, Reinigung, Dauerhaftigkeit, Abrieb, Schädlinge, Einsatz im freien vs nur drinnen
                    -Terrassenlager:
                        -->aus Kunststoff gefertigt
                        -->frost- und wasserbeständig
                    -WPC-Terrassendielen
                        -->Wood Plastic Composites
                        -->Verbundwerkstoff aus verschiedenen Gewichtsanteilen von Holzfasern(ca 50-75 Prozent) und Kunststoff(ca 25-50 Prozent)
                        -->Weitere Additive und Bindemittel:
                            -->umweltfreundlich und vollkommen unbedenklich für die Gesundheit
                            -->genügen ökologischen Anforderungen und sind recyclingfähig
                        -->beständig gegenüber tierischen und pflanzlichen Schädlingen
                        -->widerstandsfähige und kratzfeste Oberfläche, pflegeleicht
                        -->passende Unterkonstruktion notwendig
                        -->kANN ALS VERPACKUNGSMATERIAL VERWENDET WERDEN! EINE MÖGICHKEIT VON MIR :)
                            -->Beimischungen verändern Eigeschaften: 
                                -->UV-Licht-Blocker, Binder und Farbpigmente
                                -->spezielle Materialkomposition macht lange haltbar
                                -->weder Vor- noch Nachbehandlung
                                    -->weder lasiert, geölt, gestrichen oder versiegelt
                                -->leicht(er) zu reinigen(als Holz)
                                    -->Reinigung mit Wasser oder einer milden Seifenlauge
                                -->Formstabilität
                                    -->halten starker Belastung stand
                                    -->splittert nicht
                                -->Große Witterungsbeständigkeit, da resistent gegen         Frost, Hitze, Pilz-, Insekten- und Termitenbefall
                                -->Umweltfreundlich, recyclingfähig
                                -->Bei der Herstellung werden ausschließlich erneuerbare Materialien verwendet
                                -->Holzantel kann über 70 prozent gehen
                                    -->Holz aus heimischen regionen
                                -->Altholz kann verwendet werden, das sich aufgrund von Sturmschäden oder ihrer Form nicht für den Möbelbau(oder andere bereiche) empfiehlt
                                -->lange lebensdauer
                                -->Nachteil: verfärbungen, beschädigung möglich, feuchteschäden
                    -(Zier)Kies und -Split
                        -->Eigenschaften von Kies und Splitt
                            -->DIN 18196: gibt Größe der Körner an
                                -->Zierkies: Korndurchmesser meist 2-63mm
                                -->Ziersplit: Korndurchmesser meist 2-32mm
                        -->Kies ist rund, Split ist kantig, eckig, gebrochen
                                -->%kann sich auf die transporteigenschaften auswirken
                        -->Basis ist immer Naturstein
                        -->Material: Marmorkies, Kies aus Quarz oder Granit,Basalt,         Granit, Splitt aus Schiefer, Lavastein 
                        -->witterungsbeständigkeit
                        -->zu kaufen in folgenden Möglichkeiten:
                            -->20 kg Sack
                            -->Big Bags zu 500 und 1.000 kg
                            -->Standardkörnung liegt bei +/- 8/16 mm
                            
                -Holz: -verschiedene Arten:%https://www.raabkarcher.de/sortiment/holz/
                            -->Kantholz, Rundholz, Bauschnittholz, Holzplatten oder           Holzbalken
                            -->Konstruktionsvollholz (KVH), Schnittholz, Schalungen und       Bretter sowie Brettschichtholz (BSH) und Hobelware
                            -->Holzkonstruktionen (Dachstühle; Nagelbinder,                   Holzrahmenbau), Holzwerkstoffe, Holzfaserdämmstoffe,          Terrassenholz und Balkonbeläge
                            -->Konstruktionsvollholz (KVH), Brettschichtholz (BSH),           Hobelware, Sperrholz, Terrassen-Beläge, OSB-Platten,          Spanplatten,  Holzdämmstoffe, sowie ein umfangreiches         Sortiment an sägefrischem Schnittholz
                        -Belieferung: 
                            " Die Belieferung von Kunden und Baustellen ist sowohl im Rahmen des neuen !Logistik-Konzepts SPRINT! als auch über die  Zentrallager-LKWs mit Mitnahmestapler möglich."
                                -->Logistik Konzept sprint ist ein hyperlink
                        -Standards FSC®- und PEFC™ Zertifizierung:
                            -FSC®-Zertifikat:
                                zielt mit seinen Standards auf eine umfassende Nachhaltigkeit ab, zu der neben ökologischen Kriterien wie Pestizid-Verzicht und Naturverjüngung auch ethische und soziale Aspekte zählen
                            -PEFC™-Zertifizierung:
                                bietet Ihnen ebenfalls die Gewähr, dass das Holz aus kontrolliert nachhaltiger Waldwirtschaft stammt
                                
                -Parkett, Laminat und Vinyl-Boden:
                %https://www.raabkarcher.de/sortiment/parkett-laminat-vinyl/
                    -tragfähiges Fundament besonders wichtig
                    -Parkett:
                        -->Naturprodukt aus Holz
                        -->!langlebig!, ökonomisch, hygienisch, anti-allergisch
                        -->lässt sich leicht säubern & erneuern
                        -->Arten von Parkett: 
                            -->Massivparkett  
                            -->Fertigparkett (bzw. Mehrschichtparkett)
                        -->zusätzlich häufig: lackiert, geölt, versiegelt
                        
                        -->FSC®- und PEFC™-zertifiziertes Holz(nachhaltiger Waldbewirtschaftung)
                            -->FSC®-Zertifikat:
                                -->zielt auf eine umfassende Nachhaltigkeit ab
                                -->ökologischen Kriterien:
                                    -->Pestizid-Verzicht 
                                    -->Naturverjüngung
                                -->ethnische und soziale Aspekte
                            -->PEFC™-Zertifizierung:
                                -->Holz aus kontrolliert nachhaltiger Waldwirtschaft
                        -->Landhausdielen Parkett: 
                            -->aus Massivholz hergestellt
                            -->lässt sich oft abschleifen
                            -->hohe belastbarkeit
                            -->Es gibt zwei- oder dreischichtig aufgebaute Landhausdielen
                                -->besteht aus einer Nutzschicht aus Holz sowie aus einer Trägerschicht (bei dreischichtigen noch mit Rückzugfurnier
                    -Laminat:
                        -->besteht aus einer Dekorschicht, die durch eine strapazierfähige Laufschicht geschützt wird
                        -->leicht zu reinigen
                        -->Landhausdielen Laminat:
                            -->sehr strapazierfähigen Aufbau
                        -->Oberfläche ist dicht, abriebresistent und schlagfest, strapazierfähig, hitze- und lichtbeständig und recht preiswert
                        -->Formstabilität und gute Feuchtigkeitsabsperrung
                    - Vinylboden:(=PVC-Boden=Polyvinylchlorid
                        -->unterscheidung zwischen
                            -->Vinyl auf Trägerplatten: 
                                -->sind mehrschichtig aufgebaut
                                -->bestehen unter anderem aus einer Nutz- und einer Dekorschicht, sowie einer MDF- oder HDF-Trägerplatte
                            -->massives Vinyl:
                                -->ohne Verbindung mit dem Untergrund, d.h. schwimmend, verlegt und durch eine hochwertige PU vergütete Oberfläche gekennzeichnet
                        -->wasserdichten und Schmutz abweisenden Eigenschaften, wärmespeicherung, gute schalldämmfähigkeit, leicht in der Handhabung für nicht-profis
                        -->besonders robust, kratzfest und unempfindlich gegenüber Verschmutzungen, enorm belastbar, feuchtigkeitsbeständig
                        -->pflegeleicht und auch für Allergiker geeignet, hohe lebensdauer
                        -->keine gefährliche Weichmacher, Schwermetalle oder Blei
                     -Massivholzdielen / Landhausdielen
                            -->unterscheiden sich in großflächigen Formaten und der breite
                -Rohbau:
                    -Baustoffe
                    -Putze
                    -Wärmedämmung
                -Tiefbau:
                    -Versorgung
                        -->von Duktil- und Kunststoffrohrsystemen samt passenden Formteilen, Armaturen und Hydranten über Kabelschutzrohrsysteme und Kabelschächte bis hin zu Wasserzählerschacht-Systemen
                    -Entsorgung
                        -->Entsorgung und Ableitung von Schmutz- und Regenwasser
                            -->Beton-, Steinzeug- und Kunststoffrohrsysteme für Kanäle und Schächte, Entwässerungssysteme für die Bereiche Drainage, Versickerung und Rinnen sowie Schwerer Kanalguss
                        -->Entsorgung von Abwasser oder Giftstoffen
                    -Oberfläche
                        -->Gestaltung von Straßen und Plätzen, Höfen und Gärten
                        --> Befestigung von Plätzen und Wegen im Infrastrukturbereich sowie im Garten- und Landschaftsbau
                -Türen, Tore & Fenster:
                    -Türen:
                        -->Innentüren
                            -->im Kern meist aus holz
                            -->Funktionstüren:
                                -->Zusatzleistungen wie Brand-, Rauch- und Schallschutz,  sowie spezielle Feuchtraumtüren und einbruchshemmende Türen
                            -->Furniertüren
                            -->Glastüren
                                -->Meist als glasausschnitt in anderen holztüren("Fenster")
                                -->Pflegeleichtigkeit und Feuchtigkeitsresistenz
                    -Tore:
                        -->können hochisolierend, verglast oder blickdicht sein
                        -->verschiedener Tor-Arten:
                            –->Schwing- und Sektionaltoren (für die Garage)
                            -->flügeltore
                            -->Industrietore
                    -Fenster:
                        -Details wie Größe, Form, Farbe und Rahmen-Material ist auch die Energieeffizienz ein wichtiges Thema
                        -wichtig:
                            -->gute wärmeisolierung und keine wärmebrücken
                            -->Einbruchschutz
                        -->Fenster mit Dreh-Kipp-Beschlag
                        -->Fenstertüren, die häufig auch als Schiebe- oder Faltfenster angeboten werden
                        -->Schwing und Klappschwingfenster
                        -Fensterrahmen:
                            -->aus Kunststoff, Holz, Aluminium oder aus Kombinationen dieser Materialien
                                -->Kunststoff: gute Wärmedämmung, sind pflegeleicht und halten sehr lange
                                -->Holz: 
                                    -->intensive Pflege
                                    -->müssen regelmäßig lackiert werden, haben jedoch eine gute natürliche Wärmedämmung
                                -->Aluminium: 
                                    -->keine gute Wärmedämmung
                                        -->aber: Rahmenprofile mit Dämmstoffen füllen
                                    -->Langlebig
                                -->Fensterdichtungen
                                
-----------------------------------------------------------------------------------------             
        Ideen dazu:                        
                                
        -Unterscheidung in zb Rohbau und andere Planungsphasen , in denen verschiedene Gewerke am start sind und deshalb auch verschiedene Materialien bzw Bauelemente geliefert werden.
        
        -Unterscheidung und grobe Kategorisierung von zb fluiden und feststoffen, von Baumaterialien und fertigen Bauelementen und deren Anlieferung
        
        -Unterscheidung in Dinge die zum Bau gebraucht werden zb baumaterialien(zb sand oder beton) und dinge die nur vorrübergehend oder nr ein mal gebraucht werden.
        
        -unterscheidung und überlegung welche Arten von verpackungen all diese verschiednen Baumaterialien und fertigeile haben. nur eine Folie? oder besondere sichereitsvorkehrungen (zb bei empfindlichen bauteilen)? zb einsatz von styropor
        
        -wichtig ist wirlklich eine kategorisierung und daraus dann eine generelle übersicht was man besser machen kann und daraus dann schauen welche optionen sich bieten und wie man diese auch gesetzlich durchsetzen kann.
                                
        
        
        
    Unterteilung von verschiedenen "standartisierten" Verpackungsmöglichkeiten:
        
            Meine Einschätzung:(schau doch einfach nach junge + mach bilder!)
                -Schüttgut wie zb mutterbodem, steine, kies etc
                
                
------------------------------------------------------------------------------------------

bilder aus dem raab kercher als beispielbilder etc einfügen



\ac{zb}
%dies ist die zitierweise. muss ich nochnal nachschauen inwiefern ich das angeben msus. also als fißnote oder nur im Abkürzungsverzeichnis


------------------------------------------------------------------------------------------


Fragen an Daniel Wagner:


-Inwiefern werden bereits Mehrwegverpackungen im Bauwesen eingesetzt?
       -in welchen Bereichen bereits viel, in welchen weniger?
       
    Antwort:   
       -aus der sicht eines bauunternehmers im bereich rohbau
            -europaletten
            -pfandpaletten
            -verschiedene paletten für verschiedene steine. mit pfandsystem
            -gitterboxen/euroboxen für rohre, loses material,für kleinere Steine.
            -big pack's? überwiegend schüttmterial- kies, sand, ziersplit.
                -warum keine container?
                    -wegen kosten für die anfahrt, abfahrt und miete und der größe, da containerhäufig zu groß sind. Noch größere mengen werden sonst auf dem lkw oder dem container geliefert.
                -werden jedoch nicht immer wiederverwendet.
                -wird meist mit dem Kran angehoben, oder stapler.(meist jedoch Kran)
                    -beim einfamilen haus und mehrfamilien hausbau ist immer ein kran da. es wird fast alles meist mit dem lkw geliefert, die einen kran am start aben und das dann abladen.     
            -klebeschaum, dämmschaum muss wieder abgeben werden bzw die lehren Behälter davon. ist jedoch kostenlos. 
            
-Weshalb werden nur so wenige Eingesetzt? 
        -noch nie davon gehört? (Angebot gering?), mehr Aufwand?
                -Wenn es mehr Angebote gäbe, würdest du sie mal ausprobieren?
      Antwort:
        -geringe kostengünstige Alternative
                -es ist eben einfach billiger und einfacher umzusetzen und weniger Aufwand als ein Mehrwegsystem einzurichten und Aufrecht zu erhalten.
                -Ihm persönlich ist eine Wiederverwertung wichtig, aber er weiß, dass fast alles verbrannt wir an Einwegmüll.
                -Seine Meinung: Der Verpackungsindustrie geht es nur ums Geld. Abfallindustrie ist vom Staat. Er glaubt, der Staat ist nicht daran interessiert, da sie dann weniger Geld verdienen.
            

-in welchen Bereichen könnte man sie einsetzen? (Eignung für typische Produkte (4.2))
        -wenn das gesetz alles zulässt
        -bsp wiederverwendbare traggurte, dämmmaterialien 
            -dinge mit immer gleichem maß, standardsachen die man immer braucht, egal welche baustelle

-Was muss sich deiner Meinung nch ändern um Mehrweg mehr in den Fokus zu rücken?(faktoren)
        -gesetzliche vorgaben?
        -persönliches Bewusstsein der Bauherren
        .preislich muss es sich lohnen?
    Antwort:
        --Kosteneffizienz
        --Aufwand: Muss er sich um viel kümmern? 
            -Wenn er sich um noch mehr kümmern muss, braucht das zusätzliche Zeit und evtl Personal und damit kostet es Geld
            -Beispiel schaumdose --> kein problem
                -da sie klein ist und man sie "mal eben" wegbringen/mitnehmen kann, wenn man sowieso wieder einkaufen geht.
    
        
-Wer hat überhaupt die Möglichkeien, das umzusetzen? (evtl abhängig vom gesetz)
        -Kunde?
        -Bauherr?
        -Gewerke?
        
-Andere Gewerke, die nach dem Rohbau noch zum schlüsselfertigen Bauzustand notwendig sind:
        -Estrichverleger anfragen wegen viel dämmung.
    -Gipser muss auch Dämmung anbringen
    -Maler hat viele Eimer
    -Zimmermänner nicht viel, eher wenig Abfälle, da sie überwiegend mit holz Arbeiten
    -Installateuere haben viel Kartonmüll/Kartonage
    -Elektriker haben kabelabfälle
    -Gipser muss auch noch dämmen

------------------------------------------------------------------------------------------
17:30 bis 18:05 gespräch mit Daniel Wagner
15:15 bis 16:00 bei E.Wertheimer
------------------------------------------------------------------------------------------
rdm zeug von Daniel Wagner:
     -dämmung: immer in der plastikfolie: zu schutz vom material
    -holz kommt immer mit kleinen holzbalken dazwischen: gelten als sondermüll. kostenpflichtet zu entsorgen.
        -vorgabe von der verdorgungsbetriebe. du musst es bei dir abgeben. 
    
            
          --es gbt keine pauschale lösung für alles.
          --wichtog für ihn, ist eine verpackungsmöglichkeit, die man zurückgeben kann und es stoffkreislauf entsteht, sodass es wieder verwerted wird.
                    -er glaubt, dass alles verbrannt wird.
                    - "du wirst als Unternehmer allein gelassen mit dem Müll."


------------------------------------------------------------------------------------------
-----------------------------------------------------------------------------------------
Grobstruktur der Bachelorabeit inhaltlich:
        
        Einleitung:
        
            Definitionen: Hier werden wihtige, für die Referenzierung notwendigr Grundlage erklärt. Da dies bisher die Einwegverpackungen und Mehrwegverpackungen sind, werden diese hier beschrieben. Da das Wort Verpackungen nc nicht definiert wurde, wird das hier ebenfalls getan.
        
            Man könne hier noch die beiden begriffe Baustoffe und Bauprodukte nennen, da die vor allem häufig in der Aufgabenstellung vorkommen und deshalb wichtig als Abgrenzung dienen.
        
        aktueller Praxisstand:
            
            hier werden notwendige Zahlen und der aktuelle Stand der Lage geschildert, um dem leser zu zeigen weshalb es nötig ist, nach Alternativen von Einwegverpackungen zu suchen(sollte jedoch eigentlich in die Einleitung kommen. Das muss ich noch austesten) Hierbei fehlen mir noch aktuelle Zahlen zu den Mengen an tatsächluch anfallendem Einwegverpackunsgmüll(hier kann ich evtl auf die Seite der Abfallbehöde schauen , die sollten das gut wissen, die schwierigkeit besteht hierbei jedoch darin die allein aus der aubranche stammenden Abfallmengen zu isolieren.
            Vor und Nachteile von Einweg und Mehrwegverpackunge werden genannt. Hierbei sollte? hervorgehen dass ein verzicht auf einwegverpackungen nicht möglich ist, sowohl gesetzlich als auch sinnlich, da zb ein togo becher nicht wieder zurückgegeben werden kann. andererseits ist ein togo becher keine bauindustielle Ressource.
            Danach solen die Möglichkeiten des gesetzliche rahmen genannt werden um zu klären ob und inwiefern es überhaupt möglich ist, Einwegverpackungen durch Mehrwegverpackungen zu ersetzen.
            Nutzbarkeit von anderen bereits bestehenden System , die nicht aus dem Bauwesen stammen kann zeigen , dass es in anderen branchen auch klappt, mehrwegverpackungen einzusetzen und das sich diese auch durchsetzen. Dafür fehlen jedoch noch die gründe, was uns zu meinem nöchsten kapitel führt: den Kriterien, die eine Mehrwegverpackung besser machen kann/macht, als eine Einwegverpackung.
        
        Eignungskriterien und die generelle Eignung von Mehrwegverpackungen für Bauprodukte
        
            Auf Basis der Eignungskriterien für die Verwendung von Mehrwegverpackungen von Bauprodukten und Baustoffen, kann die Analyse für Mehrwegverpackungen für typischer Bauprodukte erfolgen, sowie das Potential und die Herauforderung jener Verpackungen bestimmt werden. 
            
    -Kategorisierung von "typischen Bauprodukten und Baustoffen"
            -Bereichsweise Kategorisieren?
            -einzeln nennen?
            -Einzelne Stoffe?
                -Dämmung
                -
            -nach Material des Bauproduktes unterscheiden? 
    
--------------------------------------------------------------------------------------    
    
    -Unterscheidung zwischen Bauprodukten und Baustoffen nachfragen, denn nach wikipedia sind es die selben begriffe
        
        
    -Eigenes Kapitel für die Interviews führen?
    
    -deutsvhe Nationalbibliothek wichtig?
    
    -Lebenszyklus und Kosten:
        -bei Plastik: recycling möglich?
        
    -Operator Identifikation/idetifizieren:
            
    -Bereiche des Bauwesens aufzählen, die ich in dieser Arbeit behandle(alle) und die Unterschiede aber vor allem die geminsamkeiten der gleichen nutzung von Mehrwegverpackungssystemen zeigen. Zur vollständigkeit wichtig. falls ich nicht in jedem bereich ein experteninterview führen kann, dann sollte ich zumindest eine gute Recherche dazu anstellen. falls es bereiche im bauwesen gibt, bei denen sich eine betrachtung nicht lohnt(Mehrwegverpackungsnutzung), dann muss ich das auf jeden fall brründen.
    
    -Tauschsystem des Europools, die nach EN 13698-1 gekennzeichnet wird.
    
    EPAl und VerpackG sind bisher häufig auftretende Regularien für Verpackungsmaterial
    
    -Kontrolle bei Mehrwegverpackungen fehlt noch
        --> ob die "noch gut sind"
                -Die Prüfung erfolgt im Stichprobenverfahren (Einteilung in kritische, Haupt- und Nebenfehler), welches in der Norm genau festgelegt ist, und umfasst die Abmessungen, den Zusammenbau, die Bodenbeschaffenheit, den Anstrich und den Werkstoff der Gitterbox.

 -Lagerung, Handhabung, Transport (TUL)= Transport, Umschlag, Lagerung
 
 -Kann man:(frage)
    -be- und entladen generell in:
        -Kran
        -Kran am LKW
        (-Hubwagen nur selten)
     -Transport und Lagerung:
        -meistens mit LKW(ist das so)?
        -Medium: Palette
        
-Packgüter als Ware und Packmittel(zb banane)
        
----------------------------------------------------------------------------------------

%https://perinorm-s.redi-bw.de/perinorm/fulltext.ashx?fulltextid=521b3713a58c47dc9ce14079990e6852&userid=1a37bae9-63b1-4cef-8afa-a5709d199868 bzw DIN 30800-2 (1993-07-00)

-Schüttguteigenschaften
    -Die Eigenschaften der Schüttgüter sind entscheidend für die
Auswahl der Wagenart für ihren Transport.
Charakteristische Eigenschaften sind;
— Schüttdichte.
— Korngröße,
— Fließ-, Riesel-, Rutschverhalten,
— korrosives Verhalten,
— abrasives Verhalten,
— Verhaltensänderung durch Feuchtigkeit,
— Nässeempfindlichkeit,
— Staubbildung,
— Abwehungsempfindlichkeit,
— temperaturabhängiges Verhalten,
— Temperaturempfindlichkeit und
— Verschmutzungsempfindlichkeit.

    -wichtigsten Kriterien eines Schüttgutwagen
        -— Schutz des Schüttgutes zur Erhaltung seiner Qualität
(z.B. Schutz vor Nässe),
— Schutz des Schüttgutwagens vor negativen Ladegut¬
einflüssen (z. B. Korrosion) und
— Sicherstellung der Restlosentleerung der Schütt¬
gutwagen.
    -zusätzlich:
        -Kriterien der Wirtschaftlichkeit, z.B. Vollauslastung der Schüttgutwagen, als auch Belange des Umweltschutzes, z.B. hinsichtlich Staubemissionen, zu berücksichtigen.
    
----------------------------------------------------------------------------------------
    
VDI 4460 (2003-03-00) März 2003 --- Inhaltlich überprüft und unverändert weiterhin gültig: Dezember 2012
Mehrwegtransportverpackungen(MTV) und Mehrwegsysteme zum rationellen Lastentransport
%https://perinorm-s.redi-bw.de/perinorm/fulltext.ashx?fulltextid=20ecb68d7b544a98b63cff8991485f05&userid=1a37bae9-63b1-4cef-8afa-a5709d199868

Es geht hier UM TRANSPORT-Verpackungen:
    Die Bewertung kann in einen
    -technischen
    -ökologischen sowie
    -ökonomischen
    Bereich gegliedert werden, wobei diese Einzelbereiche einander beeinflussen können

    -Häufig ist es erforderlich, nicht nur MTV-Alternativen untereinander zu vergleichen, sondern eine oder mehrere Einweg-Alternativen mitzubetrachten
    
    -Anforderungsprofil einer MTV ässt sich aus den Funktionen ermitteln, welche die MTV in den verschiedenen Bereichen zu erfüllen hat
        • Schutzfunktionen
        • Materialflussfunktionen
        • Umwelt- und Verwendungsfunktionen
        • Marketingfunktionen
        • Identifizierungs-/Informationsfunktionen
        • Produktionsfunktionen
    
     Die MTV haben – wie alle Verpackungen – das Packgut vor äußeren Einflüssen zu schützen. Lediglich beim Einsatz der MTV für Gefahrgut findet hier eine Umkehr der Wirkung der Schutzfunktion statt. Daraus ergibt sich die Anforderung an die MTV, den Belastungen zu widerstehen, die die Umwelt über den gesamten Einsatzzeitraum auf sie ausübt.
     
-Anforderungen:
    -Schutzfunktionen:
        -Temperaturbeständigkeit des Werkstoffes
            -Allgemein nehmen die zulässigen Belastungen, insbesondere bei Kunststoffen, mit zunehmender Temperatur ab . Bei länger andauernden statischen(mechanischen) Belastungen können beispielsweise durch Kriechen des Werkstoffes irreversible Schäden an der MTV auftreten.
        -Feuchteeinwirkungen (z.B. durch Tau, Reif, Spritzwasser)
            -Belastung bei hygroskopischen Werkstoffen, z.B. Wellpappe oder Holz, zu einer Beeinträchtigung der mechanischen Kennwerte führen und ein Versagen einzelner Funktionen der MTV verursachen kann.
        -UV-Bestrahlung:
            -die Oberflächen von MTV können verwittern oder ausbleichen. Teilweise können auch Materialversprödungen auftreten, welche die mechanische Belastbarkeit einschränken
        -chemische Reaktionen bzw. Lösungsmitteleintrag
            -Packgüter(also das was in der Verpackung ist) darf nicht mit dem Packmitel reagieren.
        -Schutz vor biologischen Veränderungen, z.B. durch den Eintrag von Keimen
        -Bei verschiedenen Packgütern kann eine Durchlüftung des Packgutes gewünscht sein, andere Packgüter sind dagegen staubdicht zu halten (z.B. elektronische Bauteile)
    -Materialflussfunktionen und Schutzfunktionen    
        -Mengenerhaltung:
            -Die Ausführung der Außenwände und des Bodens sowie die konstruktive Gestaltung mit einer beispielsweise um- oder abschließenden Funktion und der Werkstoff der MTV sind bei der Anforderung zur Mengenerhaltung des Packgutes von Bedeutung.
        -Formstabilität / in ihren Abmessungen nicht veränderbare MTV
            -damit die aufeinander abgestimmten Elemente der Transportkette problemlos durchlaufen können.
        -Standhaltung oder sogar Abdämpfung von MTV,(stoßfest/-dämpfend-->schwingungsdämpfend, reißfes
            -zusätzlicher Schutz der Packgüter
        -Korrosionsschutz (va bei neschädigtem stahl)
            -indem Fremdstoffe in das Packgut einbringen oder deren Oberfläche unerlaubt verschmutzen. Darüber hinaus kann die Funktion der MTV oder Teile davon durch einen Korrosionsprozess beeinträchtigt werden, was zu einem vorzeitigen Versagen der MTV führen kann
        -verschließbaren bzw. verplombbare Öffnungen
            -gegen Diebstahl
            -vor allem bei außer betrieblichen Transport
        -geringe Entflammbarkeit
            -für zusätzlichen Schutz
            -durch die Brennbarkeit der MTV wird vor allem die jeweilige Einstufung der Brandschutzklasse des Lagers bestimmt, welche einen wesentlichen Einfluss auf die Kosten der Brandschutzversicherung ausübt
    -Materialflussfunktionen
        -Mindestumlaufzahl
            -für gute Wirtschaftlichkeit
            -ohne ihre wesentlichen Eigenschaften, beispielsweise den Schutz des Packgutes oder ihre Stapelfähigkeit, zu verlieren
        -Innenabmessungen/Fassungsvolumen
            -bestimmen die Größe der MTV
            -Außenabmessungen der MTV oder gegebenenfalls des Packgutes sollten Modulmaße nach DIN 55 510 sein, um die vorhandenen Mittel der Transportkette optimal ausnutzen zu können.
        -Stapelfähigkeit
            -Die Säulenstapelung ist zu erfüllen(bzw Modulstapelfähigkeit bei unförmig)
            -Formschlüssige Verbindung miteinander verringert den Aufwand zur Ladeeinheitensicherung
        -Einfache Handhabung (Tragfähigkeit)
            -bei Manueller Aufnahme: Griffe oder Durchbrüche
            -bei mechanischer Aufnahme: entsprechenden Aufnahmeeinrichtungen
                -->geeignete Unterfahrbarkeit bzw Möglichkeit des Kraneinsatzes
        -volumenreduzierbare MTV für Rücktransport
            -geringer Aufwand und Kostenersparnis bei großer Anzahl von MTV über längere Entfernungen
        -Reparaturfähigkeit von MTV aus Kunststoff durch den Austausch einzelner beschädigter Elemente (das Unterscheidet eine MTV sehr von einer Einwegverpack)
            -Reduzierung von Abfallmengen und gegebenenfalls auch die Kosten
            -nicht mehr zu verwendenden Teile oder MTV 
                -->roh- oder werkstofflich Verwertung
                
    -Umwelt- bzw. Verwendungsfunktionen und Marketingfunktionen
        -Reinigungsfähigkeit der MTV
            -Aufgrund hygienischer Vorschriften (z.B. FLC) oder aus Marketinggründen
        -entsprechenden Kennzeichnungen,die beispielsweise die Packgutdaten oder die Versanddaten enthalten    
        -genormt
        -wiederverwertbar
        -hygienisch unbedenklich
    -Identifizierungs-/Informationsfunktionen und Marketingfunktionen
        -werbend, informativ, identifizierbar, unterscheidbar
    -Produktionsfunktionen:
        -positionierfähig, maschinelle Entnahme der Packgüter, einsatzfähig in automatisierten Prozessen
        
        
    !!!!!!!!!!    
    Nach der Ermittlung der Eigenschaften der MTV und der Auswahl verschiedener in Betracht kommender MTV ist ihre EIGNUNG für den speziellen Anwendungsfall zu bewerten. Ein mögliches Verfahren für eine Auswahl anhand qualitativer Kriterien stellt die Nutzwertanalyse dar, die in der vorhandenen Fachliteratur ausführlich beschrieben ist.
    !!!!!!!!!!        
    
    -qualitative Kriterien: 
    
    -Nutzwertanalyse
    %https://studienretter.de/nutzwertanalyse/
        -Die Nutzwertanalyse ist ein Instrument zur Bewertung von Alternativen gemäß verschiedenen Zielkriterien. Die Nutzwertanalyse ist in der Lage eine größere Anzahl Entscheidungsalternativen aufgrund gegebener Kriterien zu bewerten und im Sinne der Präferenzen des Entscheidungsträgers zu ordnen.
        - Ein mögliches Verfahren für eine Auswahl anhand !qualitativer! Kriterien
        %Quelle Ende
    
    -Qualitätsgesicherte Mehrwegladungsträger
        -Ein Beispiel für einen qualitätsgesicherten Ladungsträger ist die EUR-Flachpalette. Unter dem Dachverband der EPAL (European Pallet Association, Münster) wird durch den Zusammenschluss von Pool-Trägern, Herstellern, Händlern, Reparateuren und Verwendern die Qualitätssicherung von EUR-Flachpaletten (und EUR-Gitterboxen) vorgenommen.
        Alle Hersteller und Re parateure von EUR-Flachpaletten benötigen in den Mitgliedsländern der EPAL eine EPAL-Zulassung und sind verpflichtet, nur gütegesicherte Produkte in Verkehr zu bringen. Dies wird durch ständige Kontrollen neutraler Prüforganisationen gewährleistet
    
-Grundformen der Mehrwegladungsträger:
        -Hauptgruppen der Mehrwegladungsträger:
            -Mehrwegladungsträger mit tragender Funktion (z.B. Flachpaletten)
            -Mehrwegladungsträger mit tragender und umschließender Funktion (z.B. Boxpaletten, Paletten mit Aufsetzrahmen, Rungenpaletten)
            -Mehrwegladungsträger mit tragender, umschließender und abschließender Funktion (z.B. Tankpaletten, IBC)
        -Nachfolgend wird aus der Vielzahl der am Markt angebotenen Mehrwegladungsträger eine Auswahl vorgestellt.
    -Mehrwegladungsträger mit tragender Funktion Flachpalette
        -Eine Flachpalette besteht aus einer Platte, die durch Träger oder Klötze getragen wird oder aus zwei Platten, die durch Träger oder Klötze miteinander verbunden sind. Die Platte kann geschlossen sein oder aus mehreren Teilen bestehen. Als Werkstoffe werden Metall, Kunststoff und/oder Holz eingesetzt. %Hier am besten eine Definition einer Flachpalette aus einer Norm
        -dient dem Transport, dem Umschlag und der Lagerung von Gütern verschiedener Art und der Zusammenstellung zu Ladeeinheiten.
        -Zu beachtende Kriterien:
            -Regallagerfähigkeit 
            -manuelle/automatische Förderfähigkeit
            -zwei-/vierseitige Unterfahrbarbeit 
            -Tragfähigkeit in Abhängigkeit von Bauform und Werkstoff
            -Formstabilität
            -Sicherung der Ladeeinheiten
    -Mehrwegladungsträger mit tragender und umschließender Funktion
        -Boxpalette
            -hat einen Aufbau von mindestens vier festen, abnehmbaren oder abklappbaren Wänden, die vollwandig, aus einzelnen Stäben oder aus Gittern sein können und die mit oder ohne Deckel stapelbar sind. Hauptsächlich kommen Metall und Holz zum Einsatz. Innenauskleidung aus Kunststoff und/oder Pappe ist möglich. Sie dient dem Transport, dem Umschlag und der Lagerung verschiedenster, auch ungleichartiger oder ungleichgewichtiger Güter.
            -Zu beachtende Kriterien sind beispielsweise:
                -Stapelfähigkeit 
                -Formstabilität 
                -keine Volumenreduktion für Leergut bei starren, festmontierten Wänden -Zugriffsmöglichkeit auch im Stapel durch klappbare Seitenteile 
                -je nach technischer Ausführung regallagerfähig und manuelle/automatische Förderfähigkeit 
                -Ladeeinheitensicherung weitgehend vorhanden
        -Palette mit Aufsetzrahmen
            -eine Palette, auf die ein vollwandiger Rahmen, Gitterrahmen, einzelne Stäbe oder Bügel aufgesetzt werden. Es können mehrere dieser Einheiten bei entsprechend konstruktiver Ausführung übereinander gestapelt werden. Deckelabschluss ist möglich. Als Werkstoffe werden Holz, Metall und/oder Kunststoff eingesetzt. Innenauskleidungen aus Kunststoff und/oder Pappe sind je nach Ausführung möglich. Paletten mit Aufsetzrahmen dienen dem Transport, dem Umschlag und der Lagerung verschiedenster und je nach Ausführung ungleichgewichtiger und
            ungleichartiger Güter.
            -Zu beachtende Kriterien sind beispielsweise:
                -Stapelfähigkeit 
                -Bei entsprechender konstruktiver Ausführung ist durch mehrere Aufsetzrahmen eine Volumenvergrößerung unter Beachtung der Standsicherheit möglich. 
                -Regallagerfähigkeit 
                -Bei entsprechender Ausführung ist keine zusätzliche Ladeeinheitensicherung erforderlich. 
                -Volumenreduktion bei Leertransport und Lagerung • Austauschbarkeit defekter Einzelkomponenten möglich 
                -erhöhtes Verlustrisiko von Einzelkomponenten 
        -Rungenpalette:
            -mit vorzugsweise an ihren Ecken fest oder abnehmbar angeordneten Pfosten (Rungen), die das Aufsetzen einer weiteren Palette ermöglichen. Sie werden im Regelfall aus Metall gefertigt. Sie dienen beispielsweise zum Transport, dem Umschlag und der Lagerung von sperrigen, langen, druckempfindlichen oder nicht übereinander stapelfähigen Gütern. 
            -Kriterien sind beispielsweise: 
                -Stapelfähigkeit 
                -Formstabilität
                -Volumenreduktion bei abnehmbaren Rungen 
                -keine ausreichende Ladeeinheitensicherung 
                -bedingte Regallagerfähigkeit

    -Mehrwegladungsträger mit tragender, umschließender und abschließender Funktion:
        -Silo- und Tankpalette
            -ist eine geschlossene, vierwandige Boxpalette, die sich zum Transport von Schüttgut und Flüssigkeit eignet. Sie hat in der Regel oben eine verschließbare Einfüllöffnung und unten eine Entleerungseinrichtung. Sie wird vorzugsweise aus Metall/Holz und/oder Kunststoff hergestellt und für Transport, Umschlag sowie Lagerbereiche eingesetzt. 
            -Zu beachtende Kriterien sind beispielsweise: 
                -Stapelfähigkeit 
                -Regallagerfähigkeit 
                -Formstabilität 
                -je nach Ausführung für Gefahrgut geeignet 
                -Volumenreduktion/Innenauskleidung (Reinigung)
        -Intermediate Bulk Container (IBC)
            -Diese MTV sind mit einer Einfüllöffnung im Oberboden versehen. Der Abflus s befindet sich meistens an einer Seite unten und wird mit einer Armatur, z.B. einem Kugelhahn, verschlossen. 
            -Zu beachtende Kriterien sind beispielsweise: 
                -Stapelfähigkeit 
                -Beförderung von Waren durch verschiedene Verkehrsträger ohne Umladung des Gutes 
                -in der Regel im Leerzustand nicht volumenreduzierbar
                -Nutzung für große Umschlagmengen 
                -je nach Ausführung für Gefahrgüter geeignet und mit Inliner ausgekleidet
                
!-Mehrwegsysteme: (hier geht es um die Struktur (Infrastruktur) der Mehrwegsysteme
!    -Unterteilung aufgrund unterschiedlicher Aufbauorganisationen und Ablaufelement
!        -Beschreibung der Mehrwegsysteme:
!            -Mehrwegsysteme ohne Rückführlogistik
!                -Mietsystem:
                    -Der Versender mietet vom Systembetreiber eine bestimmte Anzahl von MTV an (Bild 15). Während der Mietdauer übernimmt der Versender alle anfallenden Leistungen, die im Zusammenhang mit der Rückführlogistik stehen
                        -Vorteil(gegenüber Kauf):
                            -Kapitalbindungskosten einzusparen und damit Fixkosten abzubauen
                            -Zusätzliche Anmietung bei Spitzenauslastungen
                                -eigenen Bestand nur für die Grundlast auslegen, wodurch wiederum Kapitalbindungskosten einzusparen sind
                        -Nachteil:
                            -bei länger- fristigen Einsatz kann das Anmieten von MTV allerdings auch zu höheren Kosten führen
!            -Mehrwegsysteme mit Rückführlogistik
!                -Tauschsysteme
!                    -Direkttausch
                        -Anlieferung des Vollgutes beim Empfänger die vollen gegen die leeren MTV direkt getauscht. Das Leergut wird anschließend zum Versender zurückgeführt. Kennzeichnend für dieses System ist, dass:
                            -der Empfänger die gleiche Anzahl der zu erwartenden MTV an leeren MTV bereitzuhalten hat
                            -die Anzahl der MTV zwischen beiden Partnern stets gleich bleibt und nur beschädigte MTV ausgetauscht werden 
                            -der Versender für den Zeitraum zwischen dem Vollgutversand und der Rückgabe des Leergutes eine bestimmte Anzahl an MTV für die laufende Produktion vorzuhalten hat
                            -vom Versender und Empfänger für den Fall von Schwankungen der Liefermenge MTV im Lager bereitzuhalten sind
                        -Vorteil:
                            -erfordert den geringsten organisatorischen Aufwand
                            -Kontrolle der Anzahl und der Qualität der MTV kann beispielsweise beim Entladen mit durchgef ührt und protokolliert werden, so dass eine aufwändige Bestandsführung im Allgemeinen überflüssig ist
                        -Nachteil:
                            -keine große Chance, Beanstandungen verursachergerecht zuordnen zu können. Problematisch wirken sich außerdem Schwankungen in der Produktion aus, da sich innerhalb des Systems die Anzahl der MTV im Mehrwegsystem im Allgemeinen nicht kurzfristig ändern lässt
!                    -Zug-um-Zug Tausch
                        -Miteinbeziehen eines Transportdienstleisters, der für den Vollgut- und Leerguttransport sorgt und seinerseits ebenfalls einen Teil der im System vorhandenen MTV zwischenlagert
                        -Der Transportdienstleister liefert aus seinem Lager dem Versender zum Zeitpunkt des abgehenden Vollguttransportes die gleiche Anzahl an leeren MTV an, die beim Vollguttransport versendet werden. Der Tauschvorgang beim Empfänger ents pricht dem des „Direkttausches“, wobei das Leergut zum Lager des Transportdienstleisters zurückgeführt wird
                        -Es erfolgen nacheinander mehrere einzelne Direkttauschvorgänge. Der Versender erhält für die – mit dem Vollguttransport – abgehenden MTV sofort Ersatz
                        -Nachteil:
                            -Es wird eine größere Anzahl an MTV benötigt(gegenüber dem Direkttausch")
                                - kann zu zusätzlichen Kosten führen für: 
                                    -die Kapitalbindung 
                                    -die Lagerung 
                                    -das Handling
!                    -nachträglicher Tausch
                        -Der Empfänger vereinbart mit dem einbezogenen Transport-Dienstleister eine Rückgabefrist, so dass er die MTV erst einen bestimmten Zeitraum nach der Anlieferung des Vollgutes auszutauschen hat (Bild 18). Der Transportdienstleister seinerseits muss erst nach Ablauf dieser Frist, wenn er die MTV von dem Empfänger zurückbekommen hat, die gleiche Anzahl an MTV an den Versender weiterleiten.
                        -Eigenschaft(da ich nicht weiß ob es ein vor der nachteil ist)
                            -der Versender hat eine größere Anzahl an MTV für die fortlaufende Produktion zu bevorraten, da er die MTV mit Verzögerung zurückerhält. 
!                -Transfersysteme
                    -Der Versender mietet eine MTV beim Systembetreiber für einen vertraglich festgelegten Zeitraum an
                    -Während des Mietzeitraumes übermittelt der Versender dem Systembetreiber bei einem vorgesehenen Vollguttransport die Versanddaten. Der Systembetreiber übernimmt lediglich den Rücktransport der beim Empfänger anfallenden entleerten und gegebenenfalls im Volumen reduzierten und zu einer Ladeeinheit zusammengefügten MTV. Der Versender erhält somit stets dieselben MTV zurück
                -Depotsysteme
                        -Es werden meist alle im Zusammenhang mit der Rückführlogistik anfallenden Aufgaben vom Systembetreiber übernommen. Dieser liefert die MTV bedarfsgerecht beim Versender an und führt das beim Empfänger angefallene Leergut wieder in Depots zurück. Dort werden die MTV für den nächsten Einsatz vorbereitet, das heißt sie werden:
                            -kontrolliert
                            -gegebenenfalls repariert oder gegen neue MTV ausgetauscht
                            -gegebenenfalls gereinigt
                            -bis zum nächsten Abruf zwischengelagert
                    -Verbuchungssystem
                        -der Versender übermittelt gleichzeitig mit dem Vollguttransport die Versanddaten zur Verfolgung der MTV an den Systembetreiber (Bild 20). Bei Mehrwegsystemen, bei denen der Systembetreiber gleichzeitig Frachtführer ist, erhält dieser die Versanddaten mit den Angaben über den Vollguttransport. Die beim Empfänger entleerten und zur Versendung vorbereiteten MTV werden anschließend nach Bedarf bzw. nach Absprache von dem Systembetreiber abgeholt und in ein Depot zurückgeführt. Die Kosten für diese Systeme werden entweder durch Miete und/oder durch ein Nutzungsentgelt für jeden Umlauf abgegolten
                    -Pfandsystem
                        - Der Systembetreiber erhält für die angelieferten MTV ein Pfand, welches anschließend beim Vollguttransport über alle Distributionsstufen weitergegeben wird. Die beim Empfänger entleerten und zur Versendung vorbereiteten MTV werden anschließend nach Bedarf bzw. nach Absprache von dem Systembetreiber abgeholt und in ein Depot zurückgeführt. Die Kosten für dieses System werden entweder durch Miete und/oder durch ein Nutzungsentgelt vom Versender an den Systembetreiber für jeden Umlauf aufgebracht. Als weitere Kosten hat der Versender die Kapitalbindungskosten für das Pfand miteinzubeziehen 
                    -Nutzerspezifische Systeme
                        -beruhen auf bilateralen Vereinbarungen zwischen dem Systembetreiber – oftmals Speditionen – und dem jeweiligen Versender (Bild 22). Hier werden insbesondere die Art der verwendeten MTV sowie die Leistungen festgelegt, welche der Systembetreiber im Zusammenhang mit der Rückführlogistik zu erbringen hat. Entsprechend den Vereinbarungen kann das System als Pfand- oder Verbuchungssystem betrieben werden. Bei einigen Systembetreibern, die gleichzeitig als Frachtführer auftreten, ist der Vollguttransport durch diesen vorgeschrieben. Bei der Ausführung als Verbuchungssystem übermittelt der Versender gleichzeitig mit dem Vollguttransport die Versanddaten zur Verfolgung der MTV an den Systembetreiber. Bei Mehrwegsystemen, bei denen der Systembetreiber gleichzeitig Frachtführer ist, erhält dieser die Versanddaten mit den Angaben über den Vollguttransport. Bei diesem Pfandsystem erhält der Systembetreiber bei Anlieferung der MTV beim Versender von diesem ein Pfand, welches über die gesamte Distributionskette entgegen der MTV-Laufrichtung weitergegeben wird. Die beim Empfänger entleerten MTV werden anschließend nach Bedarf bzw. nach Absprache von dem Systembetreiber abgeholt und in ein Depot zurückgeführt. Die Kosten für diese Mehrwegsysteme werden entweder durch eine Miete und/oder durch ein Nutzungsentgelt für jeden Umlauf abgegolten. Bei der Ausgestaltung als Pfandsystem hat der Versender zudem die Kapitalbindungskosten für das Pfand in die Kostenrechnung einzubeziehen. Darüber hinaus wird oftmals bilateral festgelegt, wer Eigentümer der MTV ist
        
---------------------------------------------------        
        
    -->Die Ausführungsformen einiger MTV werden in Normen festgelegt (z.B. „Kleinladungsträger der Automobilindustrie“, KLT-VDA (DIN 30 820) oder „Food Load Carrier“, FLC (DIN 55 423)).
    
    Die ökonomischen Aspekte bei der Einführung der MTV können nur im Gesamtzusammenhang mit der Rückführung der MTV betrachtet werden.
    
----------------------------------------------------
%https://perinorm-fr.redi-bw.de/perinorm/fulltext.ashx?fulltextid=dad1939820c645b7bcf181aab732625c&userid=bc201cca-40d9-492d-a360-14a5c25dc642
VDI 3968: Sicherung von Ladeeinheiten: Anforderungsprofil
   Definition: Ladeeinheit
        Aus einem einzelnen oder mehreren Gütern bestehendes Packgut, das als Ganzes während des Durchlaufens der Transportkette bzw. in der Warendistribution transportiert, umgeschlagen und/oder gelagert wird. 
            -Anmerkung: Man unterscheidet zwischen Ladeeinheiten mit und ohne Ladungsträger.
            
    -TUL(-Prozess):
        -Unter einem TUL-Prozess versteht die Logistik die drei Hauptprozesse:
            -Transport
            -Umschlagen 
            -Lagern 
            von Transportgütern. 
        
        -Prozessablauf in 3 Stufen
            -Prozessbeginn: Auftrag oder Bestellung;
            -Ortsveränderungen: Fortbewegung der Transportmittel durch Transport (Personentransport, Gütertransport, Tiertransport).
            -Prozessende: Lieferung, Zustellung.
    -Anforderungen an die Ladeeinheitensicherung
        -Die Anforderungen an die Ladeeinheitensicherung setzen sich aus allgemeinen und spezifischen Eigenschaften der Ladeeinheit und den bei den TUL-Vorgängen auftretenden Beanspruchungen zusammen. Die TUL-Beanspruchungen berücksichtigen die während der TUL-Vorgänge auftretenden und in Form von Belastungen auf die Ladeeinheit wirkenden Beanspruchungen. Sie müssen bei der Auswahl und Dimensionierung der Ladeeinheitensicherung berücksichtigt werden.
        -Beschaffeheit:
            -Allgemeine Eigenschaften der Ladeeinheit
                -Bedingt durch die Packgüter unterscheidet man folgende Beschaffenheiten einer Ladeeinheit:
                    -kompakt
                    -expandierend
                    -Größen verändernd
                    -verdichtbar
            -Spezifische Eigenschaften der Ladeeinheit
                -Material (Werkstoff)
                -Geometrie (Länge, Breite, Höhe, gegebenenfalls Schwerpunkt)
                -Masse
                -Oberflächenbeschaffenheit (besondere Empfindlichkeit)
                -Kantenbeschaffenheit und -ausbildung
                -Stapelbarkeit bei statischer Belastung
                -Eigenstabilität bei dynamischer Belastung
                -Temperaturempfindlichkeit
                -Feuchtigkeitsempfindlichkeit
                -sonstige Empfindlichkeiten gegenüber chemischen, biologischen oder weiteren physikalischen Beanspruchungen
        -TUL-Beanspruchungen
            -Mechanische Beanspruchungen
                -Unterscheidung in 
                    -statischen
                    -(wirken gleichermaßen auf die Ladeeinheit und auf das einzusetzende Sicherungsverfahren)
                        -Zug
                        -Druck
                        -Biegung
                        -Torsion
                        -Scherung
                    -dynamischen Beanspruchungen
                        -(bestehen aus Stößen und Schwingungen, die als Beschleunigungswerte angegeben werden und in a llen Richtungen auftreten können
            -Weitere Beanspruchungen:
                -chemische
                    -Einwirkungen durch Abgase oder Dämpfe
                -biologische
                    -Insekten, Nagetiere und Schimmelbildung
                -klimatische (physikalisch-chemische)
                    -Temperaturen und Temperaturschwankungen sowie Feuchtigkeitseinflüsse und besondere Witterungsverhältnisse wie Eis, Schnee und Hagel
                    -physikalisch: Strahlungseinflüssen
                -Staub und Schmutz
            -Zusammengesetzte Beanspruchungen
                -In aller Regel treten alle genannten Beanspruchungen nicht alleine auf, sondern überlagern und verstärken sich gegenseitig 
        -Funktion der Ladeeinheitensicherung
            -Die Funktion von Ladeeinheitensicherungen setzt sich aus den folgenden Teilfunktionen zusammen:
                -Schutz- und Sicherheit
                -Materialfluss
                -Umwelt
                    -Schutz- und Sicherheitsfunktion
                        -zur Vermeidung von Schäden an dem Packgut durch TUL-Beanspruchungen, wie sie im Beanspruchungsprofil beschrieben sind
                        -Vermeidung von:
                            -Auseinanderfallen der Ladeeinheit oder Verrutschen, Verrollen, Umkippen, Bruch, Deformation und Ausfächern einzelner Teile dieser Ladeeinheit
                            -verhindert das ungewollte Freisetzen von Gütern und damit verbundene Umweltgefährdungen
                            -bedeutet gleichzeitig eine Sicherheitsfunktion für die Beteiligten der TUL-Vorgänge (Arbeitssicherheit) sowie für unbeteiligte Personen, z. B. andere Verkehrsteilnehmer
                            -sicherstellung, dass die Ladeeinheit oder einzelne Packstücke sich nicht unbeabsichtigt öffnen oder Volumenänderungen eintreten. Weitere Schutzfunktionen betreffen klimatische, chemische und biologische Einflüsse, sowie die Vermeidung von Verlust, Diebstahl oder Verwechslung
                    -Materialflussfunktion
                        -die Eigenschaft der Ladeeinheitensicherung, als Hilfsmittel bei TUL-Prozessen oder zur Rationalisierung und Produktivitätssteigerung zum Tragen zu kommen
                        -dentifikationsfunktion
                            -für die Ladeeinheitensicherung im automatisierten Materialfluss
                    -Umwelt und Sicherheit
                        -die Möglichkeiten des Recyclings, der Wiederverwendbarkeit zur Ressourcenschonung oder zur Entsorgung
    -Anforderungsprofil der Ladeeinheitensicherung
        -Auswahl und Dimensionierung
            -Parameter mit dem Leistungsprofil der verschiedenen Sicherungstechniken
                -Leistungsprofil Untergliederung
                    -Materialeigenschaften: 
                        -Stellen die Antwort auf die Beanspruchungen dar   
                        -geben einen Überblick über die wesentlichen Werkstoffeigenschaften der jeweiligen Sicherungsmaterialien
                        -liefern Daten 
                            -zur statischen und dynamischen mechanischen Belastbarkeit (z. B. Bruchkräfte, zulässige Spannungen und Dehnungen)
                            -zur Feuchtigkeits- und Temperaturbeständigkeit und zum Schutz von biologischen, chemischen und sonstigen physikalischen Belastungen
                            -es müssen detaillierte technische Angaben bekannt sein
                    -Systemeigenschaften: 
                        -sind die Antwort auf die geforderten Funktionen der Ladeeinheitensicherung
                        -liefert Angaben über die Erfüllung von Schutz-, Materialfluss- und Umweltfunktionen
                        -Es braucht nur die prinzipielle Eignung nachgewiesen werden
                    Bsp: 
                        -Wenn also z. B. Schutz vor Schimmelbefall oder schädlichen Dämpfen gefordert  ird und diese Leistungen prinzipiell von Umreifungsbändern nicht erbracht werden können, dann stimmen die Systemeigenschaften der Umreifungsbänder nicht mit dem Anforderungsprofil an die Ladeeinheitensicherung überein.
                        
    -Auswahl von LadeeinheitenSicherungssystemen – Fallbeispiele
        -Fallbeispiel 1 – Papptonnen stehend und liegend auf Palette 
        -Fallbeispiel 2 – Volumenänderung bei Wellpappezuschnitten auf Palette
        -Fallbeispiel 3 – Kartonagen im Blockstapel auf Palette
        
        Reihenfolge ist immer:
            -Beschaffenheitsprofil
            -Beanspruchungsprofil
            -Funktionsprofil
            ----->Ladeeinheitensicherung
        
        genauere Ausführungen der Fallbeispiele schaue nochmal nach unter:
        %https://perinorm-fr.redi-bw.de/perinorm/fulltext.ashx?fulltextid=dad1939820c645b7bcf181aab732625c&userid=bc201cca-40d9-492d-a360-14a5c25dc642
    
                
    