
\chapter{Ansätze von Mehrwegverpackungssystemen zur Ersetzung von bisherigen Einwegverpackungen}
\label{sec:Ansätze}

\section{vorhandene Mehrwegverpackungssysteme}
\label{sec:Ansätze: Pfand-System}

\subsection{Mehrwegsysteme ohne Rückführlogistik}
\label{System ohne Rückführlogistik}

\subsection{Mehrwegsysteme mit Rückführlogistik}
\label{System mit Rückführlogistik}



\section{Potentiale und }
\label{sec:Ansätze von Mehrwegverpackungssystemen zur Ersetzung von bisherigen Einwegverpackungen:}

    auf schlechten Verpackungen bzw Bereichen, die viel Einwegverpackung benutzen(Kapitel 3.1) basierte Lösung
    
    


%Hier dann diejenigen Verpackungssysteme beschreiben und eine alternative finden, die ich im raab kercher gesehen habe die keine mehrwegverpackungen waren + diejenigen mehrwegverpackungssysteme, die man noch besser/effizienter hätte gestalten können

VDI 4460 (2003-03-00) März 2003 --- Inhaltlich überprüft und unverändert weiterhin gültig: Dezember 2012
Mehrwegtransportverpackungen(MTV) und Mehrwegsysteme zum rationellen Lastentransport
%https://perinorm-s.redi-bw.de/perinorm/fulltext.ashx?fulltextid=20ecb68d7b544a98b63cff8991485f05&userid=1a37bae9-63b1-4cef-8afa-a5709d199868

Achtung: hier bisher nur MehrwegTRANSPORTVerpackungen
    -Unterschied zu Lager- und Umschlagfähigen Verpackungen?

-Mehrwegsysteme: (hier geht es um die Struktur (Infrastruktur) der Mehrwegsysteme
    -Unterteilung aufgrund unterschiedlicher Aufbauorganisationen und Ablaufelement
        -Beschreibung der Mehrwegsysteme:
            -Mehrwegsysteme ohne Rückführlogistik
                -Mietsystem:
                    -Der Versender mietet vom Systembetreiber eine bestimmte Anzahl von MTV an (Bild 15). Während der Mietdauer übernimmt der Versender alle anfallenden Leistungen, die im Zusammenhang mit der Rückführlogistik stehen
                        -Vorteil(gegenüber Kauf):
                            -Kapitalbindungskosten einzusparen und damit Fixkosten abzubauen
                            -Zusätzliche Anmietung bei Spitzenauslastungen
                                -eigenen Bestand nur für die Grundlast auslegen, wodurch wiederum Kapitalbindungskosten einzusparen sind
                        -Nachteil:
                            -bei länger- fristigen Einsatz kann das Anmieten von MTV allerdings auch zu höheren Kosten führen
            -Mehrwegsysteme mit Rückführlogistik
                -Tauschsysteme
                    -Direkttausch
                        -Anlieferung des Vollgutes beim Empfänger die vollen gegen die leeren MTV direkt getauscht. Das Leergut wird anschließend zum Versender zurückgeführt. Kennzeichnend für dieses System ist, dass:
                            -der Empfänger die gleiche Anzahl der zu erwartenden MTV an leeren MTV bereitzuhalten hat
                            -die Anzahl der MTV zwischen beiden Partnern stets gleich bleibt und nur beschädigte MTV ausgetauscht werden 
                            -der Versender für den Zeitraum zwischen dem Vollgutversand und der Rückgabe des Leergutes eine bestimmte Anzahl an MTV für die laufende Produktion vorzuhalten hat
                            -vom Versender und Empfänger für den Fall von Schwankungen der Liefermenge MTV im Lager bereitzuhalten sind
                        -Vorteil:
                            -erfordert den geringsten organisatorischen Aufwand
                            -Kontrolle der Anzahl und der Qualität der MTV kann beispielsweise beim Entladen mit durchgef ührt und protokolliert werden, so dass eine aufwändige Bestandsführung im Allgemeinen überflüssig ist
                        -Nachteil:
                            -keine große Chance, Beanstandungen verursachergerecht zuordnen zu können. Problematisch wirken sich außerdem Schwankungen in der Produktion aus, da sich innerhalb des Systems die Anzahl der MTV im Mehrwegsystem im Allgemeinen nicht kurzfristig ändern lässt
                    -Zug-um-Zug Tausch
                        -Miteinbeziehen eines Transportdienstleisters, der für den Vollgut- und Leerguttransport sorgt und seinerseits ebenfalls einen Teil der im System vorhandenen MTV zwischenlagert
                        -Der Transportdienstleister liefert aus seinem Lager dem Versender zum Zeitpunkt des abgehenden Vollguttransportes die gleiche Anzahl an leeren MTV an, die beim Vollguttransport versendet werden. Der Tauschvorgang beim Empfänger ents pricht dem des „Direkttausches“, wobei das Leergut zum Lager des Transportdienstleisters zurückgeführt wird
                        -Es erfolgen nacheinander mehrere einzelne Direkttauschvorgänge. Der Versender erhält für die – mit dem Vollguttransport – abgehenden MTV sofort Ersatz
                        -Nachteil:
                            -Es wird eine größere Anzahl an MTV benötigt(gegenüber dem Direkttausch")
                                - kann zu zusätzlichen Kosten führen für: 
                                    -die Kapitalbindung 
                                    -die Lagerung 
                                    -das Handling
                    -nachträglicher Tausch
                        -Der Empfänger vereinbart mit dem einbezogenen Transport-Dienstleister eine Rückgabefrist, so dass er die MTV erst einen bestimmten Zeitraum nach der Anlieferung des Vollgutes auszutauschen hat (Bild 18). Der Transportdienstleister seinerseits muss erst nach Ablauf dieser Frist, wenn er die MTV von dem Empfänger zurückbekommen hat, die gleiche Anzahl an MTV an den Versender weiterleiten.
                        -Eigenschaft(da ich nicht weiß ob es ein vor der nachteil ist)
                            -der Versender hat eine größere Anzahl an MTV für die fortlaufende Produktion zu bevorraten, da er die MTV mit Verzögerung zurückerhält. 
                -Transfersysteme
                    -Der Versender mietet eine MTV beim Systembetreiber für einen vertraglich festgelegten Zeitraum an
                    -Während des Mietzeitraumes übermittelt der Versender dem Systembetreiber bei einem vorgesehenen Vollguttransport die Versanddaten. Der Systembetreiber übernimmt lediglich den Rücktransport der beim Empfänger anfallenden entleerten und gegebenenfalls im Volumen reduzierten und zu einer Ladeeinheit zusammengefügten MTV. Der Versender erhält somit stets dieselben MTV zurück
                -Depotsysteme
                        -Es werden meist alle im Zusammenhang mit der Rückführlogistik anfallenden Aufgaben vom Systembetreiber übernommen. Dieser liefert die MTV bedarfsgerecht beim Versender an und führt das beim Empfänger angefallene Leergut wieder in Depots zurück. Dort werden die MTV für den nächsten Einsatz vorbereitet, das heißt sie werden:
                            -kontrolliert
                            -gegebenenfalls repariert oder gegen neue MTV ausgetauscht
                            -gegebenenfalls gereinigt
                            -bis zum nächsten Abruf zwischengelagert
                    -Verbuchungssystem
                        -der Versender übermittelt gleichzeitig mit dem Vollguttransport die Versanddaten zur Verfolgung der MTV an den Systembetreiber (Bild 20). Bei Mehrwegsystemen, bei denen der Systembetreiber gleichzeitig Frachtführer ist, erhält dieser die Versanddaten mit den Angaben über den Vollguttransport. Die beim Empfänger entleerten und zur Versendung vorbereiteten MTV werden anschließend nach Bedarf bzw. nach Absprache von dem Systembetreiber abgeholt und in ein Depot zurückgeführt. Die Kosten für diese Systeme werden entweder durch Miete und/oder durch ein Nutzungsentgelt für jeden Umlauf abgegolten
                    -Pfandsystem
                        - Der Systembetreiber erhält für die angelieferten MTV ein Pfand, welches anschließend beim Vollguttransport über alle Distributionsstufen weitergegeben wird. Die beim Empfänger entleerten und zur Versendung vorbereiteten MTV werden anschließend nach Bedarf bzw. nach Absprache von dem Systembetreiber abgeholt und in ein Depot zurückgeführt. Die Kosten für dieses System werden entweder durch Miete und/oder durch ein Nutzungsentgelt vom Versender an den Systembetreiber für jeden Umlauf aufgebracht. Als weitere Kosten hat der Versender die Kapitalbindungskosten für das Pfand miteinzubeziehen 
                    -Nutzerspezifische Systeme
                        -beruhen auf bilateralen Vereinbarungen zwischen dem Systembetreiber – oftmals Speditionen – und dem jeweiligen Versender (Bild 22). Hier werden insbesondere die Art der verwendeten MTV sowie die Leistungen festgelegt, welche der Systembetreiber im Zusammenhang mit der Rückführlogistik zu erbringen hat. Entsprechend den Vereinbarungen kann das System als Pfand- oder Verbuchungssystem betrieben werden. Bei einigen Systembetreibern, die gleichzeitig als Frachtführer auftreten, ist der Vollguttransport durch diesen vorgeschrieben. Bei der Ausführung als Verbuchungssystem übermittelt der Versender gleichzeitig mit dem Vollguttransport die Versanddaten zur Verfolgung der MTV an den Systembetreiber. Bei Mehrwegsystemen, bei denen der Systembetreiber gleichzeitig Frachtführer ist, erhält dieser die Versanddaten mit den Angaben über den Vollguttransport. Bei diesem Pfandsystem erhält der Systembetreiber bei Anlieferung der MTV beim Versender von diesem ein Pfand, welches über die gesamte Distributionskette entgegen der MTV-Laufrichtung weitergegeben wird. Die beim Empfänger entleerten MTV werden anschließend nach Bedarf bzw. nach Absprache von dem Systembetreiber abgeholt und in ein Depot zurückgeführt. Die Kosten für diese Mehrwegsysteme werden entweder durch eine Miete und/oder durch ein Nutzungsentgelt für jeden Umlauf abgegolten. Bei der Ausgestaltung als Pfandsystem hat der Versender zudem die Kapitalbindungskosten für das Pfand in die Kostenrechnung einzubeziehen. Darüber hinaus wird oftmals bilateral festgelegt, wer Eigentümer der MTV ist

-->was hier noch fehlt ist der Zusammenhang der Mehrwegverpackungssystemen !zur Ersetzung! der bisherigen Einwegverpackungen
        -bzw welche Einwegverpackungen will ich damit ersetzen?


\section{Nutzbarkeit von anderen am Markt bereits verfügbaren Systemen für das Bauwesen}
\label{Nutzbarkeit von anderen am Markt bereits verfügbaren Systemen für das Bauwesen}

Teilgebiet: bereits verfügbare Systeme 
-Lebensmittelbranche (Letztvertreiber von Lebensmittelverpackungen) 
-mit Essen oder Getränken befüllte To-Go-Verpackungen -v.a. Gastronomiebetriebe (Restaurants, Cafés, Bistros, aber auch Kantinen und Cateringbetriebe)
    -ausgenommen: z.B. Imbisse, Spätkauf-Läden und Kioske, in denen insgesamt fünf Beschäftigte oder weniger arbeiten und die eine Ladenfläche von nicht mehr als 80 Quadratmetern haben 
        –>Ab 2023 dagegen können sich Verbraucherinnen und Verbraucher sicher sein, in jedem größeren Laden einen Mehrwegbecher zu erhalten. 
        –>neue Mehrwegangebotspflicht nach Paragraf 33 des Verpackungsgesetzes Mehrwegtransportverpackungen (Transport im online Handel) 
    -Probleme: Verbrauchte und verunreinigte gegenstände 
        –>Aufbereitung notwendig 
        –>Aufbereitung erfordert zusätzliche zeitliche Aufwände und verursacht Kosten 
        –>notwendige Infrastruktur für die Rücksendung und die Aufbereitung 
        –>Herstellungs- und Anschaffungskosten von Mehrwegverpackungen gegenüber Einweglösungen höher 
        –>Mehrwegverpackungen zurück in den Kreislauf 
        –>zusätzliche Versandkosten 
        –> um zusätzliche Kosten zu umgehen: 
            Nutzung von Pay-Per-Use Modellen 
                –> Anmietung wiederverwendbarer Versandboxen für eine einmalige Nutzung 
                –>umgehung der Produktentwicklung und auslagerung der Aufbereitungsaufwände –>Aufbereitung übernehmen dabei die entsprechenden Dienstleister, welche die Kapazitäten für Mehrwegverpackungen mehrerer Onlineshops bündeln können 
                –> Sollte man sich dennoch für ein Konzept entscheiden, bei dem die Aufbereitung selbst übernommen wird, lohnt sich eine Gegenüberstellung zur Packzeit. Mehrwegverpackungen lassen sich in der Regel deutlich schneller zusammenbauen und sparen in diesem Arbeitsbereich deutlich Zeit ein. Die Einsparung bei der Packzeit kompensiert die Mehraufwände für die Aufbereitung. 
    -Reduktion des Ressourcenverbrauchs und damit zu den CO2-Einsparungen 
    -Die meisten Einwegverpackungen aus Wellpappe oder Kunststoff bestehen heute bereits zu einem Großteil aus recyceltem Material. höherwertiges material, aber auch teurer. 
    -billiger wird es durch die Anzahl an Nutzungen im Produktlebenszyklus. 
    !3-R-Regel: Reduce, Reuse, Recycle! 
        –>1. option: Reduktion des Recourcenverbrauches: 
            der vielversprechendste Hebel die Reduktion des Ressourcenverbrauchs darstellt. Das ist einfach nachvollziehbar, denn wenn Produkte nicht nachgefragt werden, sinkt auch das Angebot und somit die Produktion. Ein Produkt, welches nicht hergestellt wird, erzeugt eben auch kein CO2. 
        –>2. option: Wiederverwendung(Mehrwegsystem) 
            –> Je häufiger ein Produkt verwendet wird, umso geringer ist der Einsatz von weiteren Ressourcen 
        –>3. option: Recycling –>Da im Recyclingprozess der Einsatz von Energie und neuer Rohstoffe unabdingbar ist, ist das Recycling erst an dritter Stelle zu nennen. 
        –> 1. option am besten und 3. am schlechtesten wenn es um die einsparung von co2 geht -weiteres problem: 
            zurücksendungen. mehrweg kein problem. einweg kann nicht wiederverwendet werden und sind meistens sogar zeit und energieintesiv zu entsorgen. 
            -es gilt für alle Onlineshops in deutschland, die verpackte Produkte verkaufen, sich an dualen Recycling-Systemen zu beteiligen. –>Verpackungsgesetz)
            Die Kosten dafür werden in Zukunft steigen. Mehrwegverpackungen müssen nicht an diese Recycling-Systeme bezahlen.(Verpackungsgesetz §12) 
            -die kaufentscheidung der Bevölkerung hängt auch in vielen Bereichen von dem sozialem und ökologischem Verhalten des Unternehmens ab. 
                -Beispiel: rhinopaq (als Firma die wiederverwendbare Versandboxen in einem Mehrwegmodel entwickelt) 
            -Verpackung und versand wie gewohnt 
            -rückversand über den briefkasten 
            -Verpackung enthält QR-Code, der infrmationen zb: 
                -der Standort des nähesten Briefkasten 
                -wie viele Sendungen der erhaltene rhinopaq bereits absolviert hat 
                -wie viel co2 eingespart wurde 
                -auflistung der Onlineshops, für die dieser rhinopaq bereits in nachhaltiger Mission im Einsatz war 
                –> Kunden können sich bewusst für ein unternehmen entscheiden, die das Thema Nachhaltigkeit im Onlinehandel umsetzt  
                -Rücksendung kann zusätzlich der direkte Auslöser für eine Nachbestellung sein
                
        !!!Einwegpaletten vs Mehrwegpaletten!!! (Transportpaletten) als beispiel für ein bereits existierendes System im Bauwesen bzw in der Transportbranche.
        
    Teilgebiet: Nutzbarkeit dieser Systeme