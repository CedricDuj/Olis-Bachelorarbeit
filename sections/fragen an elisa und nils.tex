

Mehrwegtransportverpackungen und Mehrwegsysteme zum rationellen Lastentransport (VDI 4460)
    -Inwiefern muss ich hier wegen Plagiaten aufpassen, da es im prinzip mein thema ist
    
Bei Normen den link für perinorm angeben oder eibnfach die stelle möglichst genau in der DIN beschreiben?

DVI als Quelle aussagekräftig genug oder muss ich dafür die referenzierten Quellen der DIN angeben?

genauere Eingliederung der "Verpackung" generell in die Logistik?
    -also mit Packmitteldefinition und Ladeeinheit, TUL-Prozesse, packmittel, Packgüter etc
    
-Beschreibung der Nutzwertanalyse nötig? falls es das ist was ich die ganze zeit mache... 
------------------------------------------------

1. Definieren Sie die Begriffe „Einwegverpackung“ sowie „Mehrwegverpackung“ im Zusammenhang von
Bauprodukten und Baustoffen.

    -Ins Definitionskapitel
    -Andere Definitionen kommen auch alle hier rein


2. Recherchieren Sie 
    2.1 den aktuellen Praxisstand von Mehrwegverpackungen sowie
            %-ist hierbei gemeint was es bereits gibt oder was quantitativ angewand wird?
    2.2 anfallendem Einwegverpackungsmüll 
            -ich denke mal quantitative Zahlen wollen sie höhren
        %        -->aus ganz deutschland? oder pro Baustelle?
        %        -->im Zusammenhang mit Kosten? oder einfach nur Zahlen zur Verdeutlichung der Lage?
                -meine Unterteilung: größe der Baustelle
    auf Baustellen und stellen Sie die dadurch verursachten Umwelteinwirkungen dar.
        %-->Frage: Inwiefern Umwelteinwirkungen?
        %    -CO2?
        %    -Generelle Umwelteiwirkungen von Einwegverpackungen oder mit Zahlen belegt?
        %        -->bzw qualitativ oder quantitativ?

    -Experteninterview von Daniel Wagner
    -Experteninterview von Verbraucherzentrale bzw Amt für Abfallwirtschaft-->fehlt noch

3. Identifizieren Sie passende Eignungskriterien (wie z.B. Ressourceneinsatz, Transportfähigkeit,
Weiterverarbeitung etc.) für die Verwendung von Mehrwegverpackungen für Bauprodukte und Baustoffe.

    %meine Frage: 
        -FÜR die Verwendung von Mehrwegverpackungen?
            -also nur positive? 
                -oder auch Nachteile nennen (von mehrwegverpackungen)?
            -oder wann es sinn macht sie einzusetzen? (zb preis, angebot, Aufwand <=> Kosten)?
                -also Eignungskriterien generell nennen und erklären für die Verwendung von Mehrwegverpackungen?
            -geht es um die Frage welche Vorteile Mehrwegverpackungen im gegensatz zu Einwegverpackungen haben und diese soll ich nun raussschreiben?
                -JA!

4. Analysieren Sie darauf aufbauend die Eignung von Mehrwegverpackungen für typische Bauprodukte und
Baustoffe und identifizieren Sie die zugehörigen Potentiale und Herausforderungen.

    -Aufbauend auf den Eignungskriterien eine Eignung von Mehrwegverpackungen für typisch Bauprodukte
    -Ist das Mehr oder weniger eine Zuordnung von Mehrwegverpackungen zu Bauprodukten? 
        -Also welche Mehrwegverpackungen sinnvoll sind um xy(ware) zu transportieren?
    -zugehörige Potentiale und Herausforderungen (für typische Bauprodukte?)
        -geht es hier darum, dass man Mehrwegverpackungen nicht für alles benutzen kann?
            -bzw wie man das schaffen kann?
            -wenn ja 
    
    -typische Bauprodukte und Baustoffe nur nennen?
    -Unterschied Bauprodukt und Baustoff? Laut Norm ist ein Baustoff ein Teil eines Bauprodukts...


5. Erarbeiten Sie Ansätze von Mehrwegverpackungssystemen, die bisherige Einwegverpackungen von
Bauprodukten und Baustoffen ersetzen könnten.


    Es geht hier um die Erarbeiten von Mehrwegverpackungs-!SYSTEMEN!
        -im zusammanhang mit logistik?
            -ich denke mal dass hier vor allem die infos aus den VDI-Richtlinien helfen(va 4460)