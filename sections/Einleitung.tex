

\chapter{Einführung}
\label{ch:Einführung}

%% -------------------
%% | Example content |
%% -------------------

\section{Motivation der Themenwahl}
\label{sec:Einführung:Motivation der Themenwahl}

Es geht um die wissenschaftliche Motivation! nicht meine eigene! 

Es ist zu erkennen, dass viele andere Bereiche der deutschen und internationalen Industrie bereits Alternativen zu Einwegverpackungen in Form von Mehrwegverpackungen gefunden haben und sich diese nicht nur durch persönliches Interesse der Geschäftsführung jener Konzerne fundiert, sondern sich die Idee von Mehrwegverpackungen und dauerhaften Alternativen zu Einwegverpackungen auch wirtschaftlich rentiert, sowie ein positives Image der Konzerne als Folge hat.

%Eine meiner Erfahrungen auf einer Baustelle zum Bau eines Einfamilienhauses führte mir nicht nur vor Augen, wie viel Plastik im Allgemeinen Anwendung in der Bauindustrie findet, sondern auch die in meinen Augen häufige Anwendung von Plastiktüten zur Verpackung von Dämmung, sowie die einmalige Nutzbarkeit von Gurten bei der Annahme und dem Anheben von Materialien aller Art durch Einsatz eines Kranes.   






\section{Ziel und Abgrenzung der Arbeit}
\label{sec:Ziel und Abgrenzung}

(Es handelt sich vorrangig um die Frage bzw die Suche nach Mehrwegverpackungen , nur geringfügig nach anderen Arten von Einwegverpackungen(Anhang))
    
Das Ziel dieser Arbeit ist es, eine oder mehrere Alternativen zu Einwegverpackungen im Bauwesen zu finden. Der Operator "Identifikation" gibt an, dass als Ziel dieser Arbeit eine oder mehrere Alternativen zu Einwegverpackungen ermittelt bzw festgestellt werden, diese jedoch (nur als theoretische Grundlage einer ersten Recherche entstanden sind und keinerlei praktische Bestätigung enthalten.)(stimmt das so?)

Inhaltlich steht hierbei die Eignung von Mehrwegverpackungen für typische Bauprodukte und
Baustoffe im Zentrum, sowie eine Erarbeitung von Ansätzen zu Mehrwegverpackungssystemen, die bisherige Einwegverpackungen von Bauprodukten und Baustoffen ersetzen könnten. 

Nicht im Fokus dieser Bachelorarbeit steht das Thema nachwachsende Rohstoffe, die durchaus eine Alternative zu konventionellen Einwegverpackungen wären, jedoch häufig keine ausreichenden Materialeigenschaften haben, um als Mehrwegverpackung angesehen zu werden. Biologisch abbaubare und anderweitig ökologisch sinnvollere Lösungen im Hinblick auf eine Einsprung von CO2 im Lebenszyklus einer Verpackung und der ÖkoBilanz fallen unter selbigen Kriterien nicht in den Fokus dieser Arbeit. (Quelle?)

Die Bereiche des Bauwesens, die für eine Lösung des Problems bzw der Frage in Betracht kommen, sind zusammen mit der Logistik, die Lagerung und ... . Gründe für die Wahl dieser Bereiche sind... . Im Kapitel "Nutzbarkeit von anderen am Markt bereits verfügbaren Systemen für das Bauwesen", wird als Ausnahme hiervon auch die Lebensmittelindustrie und die ... genannt bzw aus ihr referenziert. (Quelle)?

Zuletzt ist eine geographische und rechtliche Abgrenzung auf das Gebiet der Bundesrepublik Deutschland und die hierin geltenden Gesetze notwendig. Die betrifft vor allem das  Verpackungsgesetz(VerpackG) und ...welches dem Erreichen der europarechtlichen Zielvorgaben der Richtlinie 94/62/EG auf Bundesebene (Übernimmt?)

Abgrenzung: 
%-Abgrenzung zu nachwachsenden Rohstoffen:
%-nachwachsende Rohstoffe wie Bio-Kunststoffe (z. B. Maisstärke), Rohrzucker oder Bambus, die biologisch abbaubar sind, kommen vermehrt zum Einsatz.
%-nur Deutschland (bzw nur nach deutschem Vorgehen was Verpackungsgesetze und recycling angeht)
        -->nicht deutsche Möglichkeiten und Ideen sind evtl anders. Kann auch in den Schluss bzw. als eine weitere anschließende Bachelorarbeit-->International-->eher die Staaten anschauen, die am meisten Einwegverpackungsmüll erzeugen-->China, Indien etc.



\section{Methodik}
\label{sec:Einführung:Methodik}

%https://www.scribbr.de/methodik/methodik-schreiben/
Beantworte diese Fragen, wenn du deine Methodik schreibst:
            -ca 10 Prozent der Bachelorarbeit
    
    Welche Art von Forschung nimmst du vor?
        
        -Quantitativ
        -Qualitativ
        -beides(mein Fall)
        -Induktives Vorgehen
            -Wenn du induktiv argumentierst, führst du eine eigene Forschung durch und leitest daraus selbst eine Theorie ab
        -deduktives Vorgehen    
            -Wenn du deduktiv vorgehst, testest du mit deiner Untersuchung eine bereits vorhandene Theorie

Diese Bachelorarbeit ist sowohl qualitativ als auch quantitativ durchgeführt worden. Zur Sammlung von Informationen wurde eine Literaturrecherche durchgeführt, die ... . 

    Wie hast du deine Daten erhoben?
    
        -literaturrecherche + Felduntersuchung:
        -Literaturrecherche bzw Literaturarbeit
            -Datenbanken durchsuchen
            -Berichte analysieren
            -Daten analysieren
            -Metaanalyse durchführen
            -Sekundäranalyse durchführen
            -Diskursanalyse durchführen
            -Quantitative Inhaltsanalyse durchführen

        -quantitative Felduntersuchung:
            -Umfrage durchführen
            -Interviews durchführen
            -Experimente durchführen
            -Beobachtungen mit signifikanten Zahlen durchführen
            -Quantitative Diskursanalyse
            
Um Informationen aus erster Hand aus der Praxis zu sammeln, wurde in dieser Bachelorarbeit ein Interview geführt. 

    Welche Eigenschaften haben deine Daten?
    
    -bei Interview Part:
            -Interviewform festlegen: strukturiert, semistrukturiert, unstrukturiert
            -Zeit und Ort
            -Beschreibung der Interviews
            -%"Um die Zufriedenheit der Kundschaft in Berlin zu untersuchen, wurden am 23. August 2020 semistrukturierte Interviews mit zehn Passanten sowie Passantinnen im Alter zwischen 20 und 50 Jahren in der Nähe der Eisdiele durchgeführt. Es wurden semistrukturierte Interviews gewählt, um geordnete Antworten auf die Teilfragen zu bekommen. So wurde außerdem die Möglichkeit geschaffen auf einige Fragen genauer einzugehen, um mehr Informationen zu erhalten.
            %Die Interviews wurden in einem Raum in der Julianaschule im Julianenweg durchgeführt und dauerten etwa zehn Minuten, wobei die Interviewten nacheinander befragt wurden. Ebenso wurden die Interviews mit dem Einverständnis der zehn Teilnehmenden aufgezeichnet, um die Antworten effektiver analysieren zu können."
            -in deiner Methodik solltest du die Wahl deiner Interview-Fragen begründen
        -bei Literaturarbeit Part:
            -Daten aus:
                -Literatur
                -Datenbanken
                -Anderen Nachschlagewerken
        -Ein- und Ausschlusskriterien
            -im Methodikteil immer angeben, welche Daten du genau verwendet hast:
                -Einschlusskriterien: Welche Daten du in die Untersuchung einbeziehst
                -Ausschlusskriterien: Welche Daten du nicht in deine Untersuchung aufgenommen hast und warum
                    -%"Nur vollständig ausgefüllte Fragebogen wurden in die Untersuchung einbezogen. Des Weiteren wurde die Umfrage nur mit 18-Jährigen durchgeführt, die zum Zeitpunkt der Datenerfassung in Berlin oder Hamburg wohnten. Wenn die Teilnehmenden diese Voraussetzungen nicht erfüllten, wurden ihre Angaben nicht in der Analyse berücksichtigt, weil sie im Rahmen dieser Bachelorarbeit irrelevant sind. 
                    %Inkorrekt ausgefüllte Fragebogen, bei denen unklar war, welche Möglichkeit angekreuzt wurde, flossen nicht in die Auswertung ein, außer wenn die Teilnehmenden konnten im Nachhinein nach der korrekten Antwort gefragt werden konnten."
    
    Wie ist deine Forschung abgelaufen?
        -auf den praktischen Verlauf eingehen
            -z.B.%ob ausreichend Leute auf deine Interviewanfrage geantwortet haben, wie viele es waren, wie viele du interviewt und wann du zufriedenstellende Ergebnisse erhalten hast.
            -Fokussiere dich auf die Rolle des/der Forschenden und die Forschungssituation. Wenn z. B. die Qualität der Interviews negativ beeinflusst wurde, weil im Nebenraum eine Bandprobe stattgefunden hat, kannst du das hier angeben. Reflektiere aber nicht subjektiv den praktischen Forschungsverlauf, sondern bleibe bei der Beschreibung der Fakten.
            
    Wie hast du deine Daten analysiert?
        -Beispielantworten für quantitative Forschung:

            -Statistische Tests
            -Regressionsanalysen
            -Metaanalyse
            -Quantitative Inhaltsanalyse
            -Kombination der genannten Methoden

        -Beispielantworten für qualitative Forschung:

            -Interpretativer oder beschreibender Vergleich von Daten
            -Kodierung und Kategorienbildung von Materialien
            -Qualitative Inhaltsanalyse
            -Untersuchung anhand eines bestimmten Modells, das du in deinem theoretischen Rahmen beschrieben hast
            -Kombination der genannten Methoden

    Sind die Gütekriterien deiner Forschung erfüllt?
        -Bei quantitativer Forschung solltest du darauf eingehen, ob deine Forschung reliabel, valide und objektiv ist.
        -Für qualitative Forschung gelten die Gütekriterien der Transparenz, Reichweite und Intersubjektivität.
        -Möglichkeiten, Gütekriterien zu beschreiben:
            -in einem gesonderten Abschnitt,
            -wenn du die Daten und den Forschungsverlauf beschreibst oder
            -in der Diskussion oder während der Datenanalyse.
    
    -Zeitform der Verben
        -Normalerweise schreibst du den Methodikteil im Präteritum (‚die Zielgruppe war‘) und im Präsens (‚die Daten werden analysiert‘). Falls deine Betreuungsperson angibt, dass sie das Präsens bevorzugt, solltest du dich daran halten (‚die Zielgruppe ist‘).
        -Wenn du auf die Gütekriterien eingehst, verwendest du das Präsens (‚die Forschung ist valide, da …‘). Dabei gelten die allgemeinen Regeln für Zeitformen beim Verfassen einer Bachelorarbeit.


    -Es ist wichtig, dass deine Forschung reproduzierbar ist. Jemand, der dieselbe Untersuchung mit der gleichen Methode durchführt, sollte zu ungefähr denselben Ergebnissen kommen


\section{Aufbau der Arbeit}
\label{sec:Einführung:Aufbau der Arbeit}

Diesen Paragraph ganz am ende ausfüllen, da ich sonst alles immer abändern muss wenn ich die struktur der BA verändere

%%Die Einleitung soll den Leser in die Problematik einführen und das gewünschte Ziel der Arbeit erläutern. Hier kann die Motivation für die Arbeit und eventuell Zusammenhänge, die zur Formulierung der Aufgabe geführt haben dargestellt werden.
%%Dem Leser soll weiterhin die genaue Vorgehensweise zur Problemlösung vorgestellt werden und einen ersten Überblick über die Gliederung der Arbeit gegeben werden. Mit der Einleitung beginnt die fortlaufende Nummerierung der Seitenanzahl.
%%Für den gesamten Text gilt folgende Formatierung:
%	Schriftart: Time New Roman
%	Schriftgröße: 12
%	Zeilenabstand: 1,5
%	Nach Absatz 10pt
