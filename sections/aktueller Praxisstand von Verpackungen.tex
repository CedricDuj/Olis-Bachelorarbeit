
\chapter{aktueller Praxisstand von Verpackungen}
\label{sec:aktueller Praxisstand von Verpackungen}

\section{Aktueller Praxisstand zum Einsatz von Mehrwegverpackungen und anfallendem Einwegverpackungsmüll auf Baustellen}        
\label{sec:aktueller Praxisstand von Verpackungen:Aktueller Praxisstand zum Einsatz von Mehrwegverpackungen und anfallendem Einwegverpackungsmüll auf Baustellen}


    -Mehrwegverpackungssysteme im Bauwesen: (aus der Sicht eines Bauunternehmers-Rohbau)
    (Daniel Wagner)
        -europaletten
        -pfandpaletten
        -verschiedene paletten für verschiedene steine. mit pfandsystem
        -gitterboxen/euroboxen für rohre, loses material,für kleinere Steine.
        -big pack's? überwiegend schüttmterial- kies, sand, ziersplit.
                -warum keine container?
                    -wegen kosten für die anfahrt, abfahrt und miete und der größe, da containerhäufig zu groß sind. Noch größere mengen werden sonst auf dem lkw oder dem container geliefert.
                -werden jedoch nicht immer wiederverwendet.
        -klebeschaum, dämmschaum muss wieder abgeben werden bzw die leeren Behälter davon. ist jedoch kostenlos. 
        -wird meist mit dem Kran angehoben, oder stapler.(meist jedoch Kran)
            -beim einfamilen haus und mehrfamilien hausbau ist immer ein kran da. es wird fast alles meist mit dem lkw geliefert, die einen kran am start aben und das dann abladen.
        
        --->Ideen dazu: 
                -Pfandsystem auch auf andere Maustoffe umlenken (bzw Verpackungssysteme)
                -zu den einzelnen Punkten ggf. noch weiter recherchieren. je nach Wichtigkeit

------------------------------------------------------------------------------------------

Daniel Wagner über seinen quantitativen "Verbrauch"

        -wie viel ist bisher einwegverpackungsmüll auf deiner Baustelle? (in [t] oder %)
        
            --dämmung in m³   1.5-2 tonne styropor und styrodur (pro einfamilienhaus)
                -Kosten bei Entsorgung: Styrodur -Abfälle ca. 260-280 Euro
            --plastikabfall: 10-15m³ verpackungsfolie, bänder für steine etc
                -Kosten: 135 Euro die Tonne
            --Einwegbänder: 0.5-1 m³
                -Stahl kommt mit einwegbändern. früher war mit Draht verbunden, dann kam eine Gesetzesänderung
                    -Stahlabfälle kostenlos
                -Einmalbänder ca 4 Euro im Einkauf. 
                -beim Einkauf ca 500-600 Euro pro Einfamilienhaus+ Entsorgung 
                -früher sind die steine mit Flachstahlbändern fixiert werden. 
                    -diese waren wiederverwertbar. es ist möglich, alles zurückzugeben, kostet aber meistens mehr und landet z.B. bei Yttong auch auch der Müllhalde.
        -Größenordnung wichtig: Einfamilienhausbau vs größere Baustellen
            -seiner Meinung nach unterschiedlich? 
                    -wenn ja inwiefern?
                        --dämmung wird geringer bei größeren gebäuden. 
                        --wird aber immer gleich geliefert.
                        --steinwolle und holzwolle auf ner palette eingeschweßt in folie
                        --ca 100m" mineralwolle pro
                        --folie wichtig, da sonst wasser zieht
                        --mineralwolle/steinwolle ist feuerfest und brand. AVR.
                        --holzwole für das dach
                        --steine auf paletten mit bändern.
                        --Porenmeton(yttong) muss eingescheißt werden, weil es wasser zieht. 





------------------------------------------------------------------------------------------

Anfallender Verpackungsmüll auf der Baustelle:

Vorgang von Abfällen:   -Transport der Abfälle zu Sammelbehältern
                        -getrennte Sammlung
                        -Abtransport
                        -entsprechende Transportwege, Lagerflächen und Stellflächen
    -Kreislaufwirtschaftsgesetz (KrWG): 
        definiert Abfälle als Stoffe oder Gegenstände, deren sich der Besitzer
        entledigt, entledigen will oder entledigen muss. Das KrWG gilt insbesondere für Erzeuger und Besitzer von Abfällen, für Einsammler oder Beförderer von Abfällen sowie für Unternehmen, die Abfälle in einem Verfahren nach Anhang II KrWG entsorgen.
    Unterschieden in:
    - Abfälle zur Verwertung und
    - Abfälle zur Beseitigung.
    außerdem unterschieden in:
    -gefährliche Abfälle (wenn durch § 48 Satz 2 KrWG „durch Rechtsverordnung be-
        stimmt worden sind“
    -nicht gefährliche Abfälle
    
Oberbegriff für alle Abfälle, die bei Bauarbeiten jeglicher Art anfallen, lautet Bauabfälle. 
        -->Unterpunkt Verpackungsabfälle: 
                        "Zu Verpackungsabfällen zählen Verkaufs- und Transportverpackungen von Bau- und Bauhilfsstoffen aus unterschiedlichen Materialien wie z. B. Kunststoffen."

Umgang mit Abfällen:
    aus rechtlichen und wirtschaftlichen Gründen eine Grundtrennung des Abfalls vorgenommen werden. Die dafür notwendigen Maßnahmen hängen vom anfallenden Abfall in Art, Geometrie und Menge, vom Bauablauf und von den Platzverhältnissen auf der Baustelle ab.

Mögliche Trennung von Verpackungsmüll trennen nach:
        Papier-Verpackungen, Styropor-Verpackungen, Folien-Verpackungen, Kunststoff-Umreifungen, Kunststoffgebinde (Fässer, Kanister usw.).
    -Aus dem Bereich der Verpackungen sollten auf allen Baustellen zunächst Folien, Pappe, Papier und Papiersäcke als wesentliche Verpackungsabfälle getrennt gesammelt werden.

Wichtige Vorschriften und Regeln:

    Kreislaufwirtschaftsgesetz (KrWG)
    Abfallverzeichnisverordnung (AVV)
    Gewerbeabfallverordnung (GewAbfV)
    DIN 18 459 VOB Vergabe- und Vertragsordnung für Bauleistungen - Teil C: Allgemeine Technische Vertragsbedingungen für Bauleistungen (ATV) - Abbruch- und Rückbauarbeiten
    
    Für jede Abfallart muss in der Regel ein Behälter vorgesehen werden. Die Wahl des Sammelbehälters sollte in Absprache mit dem beauftragten Entsorger erfolgen, um den günstigsten Entsorgungsweg zu ermöglichen.

-->wichtig hierbei: Zu Verpackungsabfällen zählen nur Verkaufs- und                                        Transportverpackungen!  

--------Quelle Ende--------



Abfall:

    %(https://www.sanier.de/entsorgung/abfallarten)
    Abfallarten auf der Baustelle:
        -vier verschiedene Kategorien:  -Siedlungsmüll, 
                                    -Sonderabfälle
                                    -Wertstoffe 
                                    -Verpackungsabfälle
            --> nur zum Teil zutreffend
                -->deshalb speziellere Unterteilungen von der Berufsgenossenschaft Bau (BG    Bau)
                -->einzelnen Abfallarten haben jeweils besondere Entsorgungsregelungen
        Andere Unetrteilung:
        -nichtmineralischen Materialien der Bautätigkeit:
        -Bauschutt (besteht aus vorwiegend mineralischen Bestandteilen)
    


Entsorgung:

    %(https://www.interseroh.de/fileadmin/PDF/Broschueren_und_Informationsmaterial/Merkblatt_zur_Entsorgung_von_Transportverpackungen.pdf)
    aktueller Stand von !Entsorgung! von Transportverpackungen(eines Unternehmens(beispielhaft))
        -Verpackungsarten:  -Papier, Pappkarton, Wellpappe mit Polsterung
                            -Folien, PE-Schrumpf-, Stretch- und Luftpolsterfolien
                            -PE-Schaumstoffverpackungen (unvernetzt)
                            -PUR-Schaumstoffverpackungen
                            -Umreifungsbänder aus Kunststoff
                            -Umreifungsbänder aus Stahl
                            -Paletten und Verpackungen aus Massivholz
                            -Paletten und Verpackungen aus Holzwerkstoffen
                            -EPS (z.B. Styropor®)
                            -Papiersäcke und -beutel sowie Verbundsäcke und -beutel (Baustoffsäcke)
                            -PE- und PP-Eimer, -Dosen, -Kartuschen, -Hobbocks, -Kanister und -Fässer
                            -Weiß- und Schwarzblecheimer, -Dosen, -Kartuschen, -Hobbocks, -Kanister und -Fässer

        -Art der Behälter/Erfassung:
                            -Wechselbehälter
                            -Umleerbehälter
                            -Sammelsäcke
                            -lose Erfassung


Duales System Deutschlands(DSD) und der grüne Punkt!!!!!!!!!




Stoffkreislauf von Einwegverpackungen als Thema?
    -->(Lebenszyklus)




%https://www.mt-metallhandwerk.de/bauabfaelle-vermeiden-sortieren-wiederverwerten-27112020
Tausch- oder Restebörsen
    -bei einem Projekt übrig gebliebene:
        -Stahl-/Metallprofile, Lack-, Isolier-, Dämmmaterial- oder Folienreste
        
Was fordert die Gewerbeabfallverordnung? 
    Für folgende zehn Abfallkategorien (Fraktionen) der Bau- und Abbruchabfälle sieht die GewAbfV eine Getrenntsammlung vor: Metalle, Dämmmaterial, Kunststoffe, Baustoffe/Gips, Holz, Glas, Beton, Bitumen, Fliesen/Keramik und Ziegel. 

Hersteler fürchten durch Transport- und Lagerschäden bedingte Reklamationen
    -->deshalb verwenden sie so viel einwegverpackungen
    
Schutt, Abfall oder Sondermüll?
Da die Entsorgungskosten für Baumischabfall deutlich höher sind als für Bauschutt, werden beim Rückbau vor dem eigentlichen Abriss möglichst viele Stoffe, die nicht zum Bauschutt gehören, entfernt.
    -Zum Bauschutt gehören alle mineralischen Stoffe, wie Sand, Backsteine, Natursteine, Mörtel, Putz, Dachziegel, Beton, Zement, Estrich, Sanitärkeramik, Fliesen, Kacheln oder Feinsteinzeug. 
    -Baumischabfälle sind Gläser, Glasbausteine, Bauholz, Metall und Schrott, Gipskarton und Tapeten, Porenbeton, Kabel und Rohre, Dämmstoffe, Isolierungen, Türen und Fenster, Verpackungen und Kunststoffe. 
    -Zum Sondermüll gehören Materialien, die Schadstoffe enthalten, zum Beispiel PCB-haltiges Altholz, Lackreste, Bauschaumdosen, Schamottsteine oder Asbestzementplatten. 
    
------------------------------------------------------------------------------------------------------

VDI 4460 (2003-03-00) März 2003 - Inhaltlich überprüft und unverändert weiterhin gültig: Dezember 2012
Mehrwegtransportverpackungen(MTV) und Mehrwegsysteme zum rationellen Lastentransport
%https://perinorm-s.redi-bw.de/perinorm/fulltext.ashx?fulltextid=20ecb68d7b544a98b63cff8991485f05&userid=1a37bae9-63b1-4cef-8afa-a5709d199868

aktueller Praxisstand: (bzw möglichkeiten, die es im Mehrwegverpackungs-Bereich im bauwesen gibt)
    
    Grundformen der Mehrwegladungsträger:
        -Hauptgruppen der Mehrwegladungsträger:
            -Mehrwegladungsträger mit tragender Funktion (z.B. Flachpaletten)
            -Mehrwegladungsträger mit tragender und umschließender Funktion (z.B. Boxpaletten, Paletten mit Aufsetzrahmen, Rungenpaletten)
            -Mehrwegladungsträger mit tragender, umschließender und abschließender Funktion (z.B. Tankpaletten, IBC)


\section{Anforderungen und Eigenschaften von Verpackungen im Bauwesen}
\label{sec:Anforderungen und Eigenschaften von Verpackungen im Bauwesen}

%https://www.medewo.com/blog/de/loesungen/was-muss-eine-verpackung-leisten/
Was muss eine Verpackung leisten? Unsere Antwort: Den Wert der Ware erhalten!


\subsection{Vor- und Nachteile von Einwegverpackungen}
\label{sec:Vor- und Nachteile von Einwegverpackungen}


%https://packpart.eu/glossar/einwegverpackungen/
Vor und Nachteile von einwegverpackungen:
-Vorteile:
--> -höhere Wirtschaftlichkeit durch günstige und leichte Herstellung
    -Individualisierbarkeit des Designs
    -vereinfachte Lagermöglichkeit
    -Gewährleistung der Transportsicherheit
    -Erfüllung von gesetzlichen und hygienischen Vorschriften
            %(https://www.bmuv.de/faqs/mehrwegverpackungen/)
        -->allgemeinen Hygienevorschriften nach der VO (EG) Nr. 852/2004
        --> Auswahl bei Ausgabe von Mehrwegverpackungen auf ein geeignetes, als sicher anerkanntes Lebensmittelkontaktmaterial achten
            --> keine Verantwortung für die Eignung und Beschaffenheit des Behältnisses für Lebensmittel oder Getränke in vom Kunden mitgebrachte Dosen oder Becher
    -hohe Attraktivität für Endverbraucher durch einfache Handhabung
    (-Geringes Gewicht: spart geld bei transport und gut für unterwegs(bei flaschen))

-Nachteile:
-->schlechtere Umweltverträglichkeit
-->Ökobilanz (besseren CO2-Fussabdruck(oder ist das eigener Punkt?))


\subsection{Vor- und Nachteile von Mehrwegverpackungen}
\label{sec:Vor- und Nachteile von Mehrwegverpackungen}


Vorteile der Mehrwegverpackungen:
%(https://packpart.eu/glossar/mehrwegverpackung/)
-mehrmalige Nutzung
-(nachhaltigen Umgang mit Ressourcen(einmaliger Aufwand zur Herstellung))
-    Robustheit
    weniger Abfall, durch mehrmalige Verwendung
    ressourcenschonender als Einwegverpackungen  
    kleinerer CO2-Fußabdruck bei der Herstellung
    Nachhaltigkeit als Image des Unternehmens
-Kosten:häufig hohe Erstbeschaffung, aber mittel- bis längerfristig billiger
-Umweltfreundliche Aspekte: Müllreduzierung, Ressourcenschonung, CO2-Bilanz(Vermeidung einer Vielzahl der Strecken, die in der Produktion per LKW transportiert werden -->Produktionskette)
- deutlich höheren Schutz der Produkte, besonders bei widrigen Wetterbedingungen wie Regen oder Schnee%(https://www.ecosistant.eu/mehrwegverpackungen-vorteile-nachteile/)


Nachteile Mehrwegverpackungen %(https://packpart.eu/glossar/mehrwegverpackung/)
-eventuelle Probleme mit der Hygiene bei der Aufbereitung und Wiederbefüllung 
-bei Änderung des Verpackungsdesigns sind Flaschen mit altem Design noch lange im Umlauf


------------------------------------------------------------------------------------------

\section{Auswirkungen und Möglichkeiten im Rahmen des Verpackungsgesetzes in Bezug auf nachhaltige Veränderungen im Bauwesen}
\label{sec:Definitionen:Auswirkungen und Möglichkeiten des Verpackungsgesetzes in Bezug auf potentielle Änderungen im Bauwesen}

-Gesetzlicher Rahmen:

%(https://www.verpackungsgesetz.com/)
Verpackungsgesetz (VerpackG):
    "deutsche Umsetzung der europäischen Verpackungsrichtlinie 94/62/EG (kurz PACK) zur Regelung des Inverkehrbringens von Verpackungen sowie der Rücknahme und Verwertung von Verpackungsabfällen."
    -ersetzt seit 03.06.21 den Vorgänger "Verpackungsgesetz 1"(VerpackG1)
      -gilt nur in der Bundesrepublik Deutschland. Jedes Land der EU verfügt über seine eigene PACK-Gesetzgebung.
    -implementiert Einwegkunststoffrichtlinie und die Abfallrahmenrichtlinie
    -Verschiedene Verpackungstypen, je nach Art der Verwendung
        -Verkaufsverpackungen, Umverpackungen, Serviceverpackungen, Versandverpackungen und Transportverpackungen
        -andere Unterscheidung: -systembeteiligungspflichtige Verpackungen (B2C)
                                    -entstehen typischerweise beim privaten Endverbraucher oder vergleichbaren Anfallstellen
                                -Verpackungen aus dem gewerblichen Bereich (B2B)
        -andere Unterteilung: Materialfraktionen
                                - Papier und Karton, Kunststoffe, Glas, Eisenmetalle, Aluminium sowie Verbundwerkstoffe
%(https://www.verpackungsgesetz.com/wp-content/uploads/gesetz_verpackg_final_fassung_ab_20220101.pdf) = offizielles Gesetz zu Verpackungen:
"Gesetz über das Inverkehrbringen, die Rücknahme und die
hochwertige Verwertung von Verpackungen (Verpackungsgesetz -
VerpackG)":

- legt Anforderungen an die Produktverantwortung nach § 23 des Kreislaufwirtschaftsgesetzes
für Verpackungen fest
-Zweck: 
    -die Auswirkungen von Verpackungsabfällen auf die Umwelt zu vermeiden oder zu verringern.
    -durch Vermeidung von Verpackungsabfällen von Verpflichteten und Vorbereitung zur Wiederverwendung oder dem Recycling. 
    - zusätzliche Wertstoffe für ein hochwertiges Recycling gewinnen
    -erreichen europarechtlicher Zielvorgaben der Richtlinie 94/62/EG über Verpackungen und Verpackungsabfälle: von anfallenden Verpackungsabfällen jährlich mindestens 65 Masseprozent zu verwerten und mindestens 55 Masseprozent zu recyceln
        - Verwertung und Recycling verschiedener Materialien(Holz, Kunstoff, Metall, Glas papier/Karton) werden über die kommenden Jaher noch steigen
-Anwendung: 
    -gilt für alle Verpackungen
-Begriff "Verpackung":
    -" Verpackungen sind aus beliebigen Materialien hergestellte Erzeugnisse zur Aufnahme, zum Schutz, zur Handhabung, zur Lieferung oder zur Darbietung von Waren, die vom Rohstoff bis zum Verarbeitungserzeugnis reichen können, vom Hersteller an den Vertreiber oder Endverbraucher weitergegeben werden"
    -Mehrwegverpackung: " Mehrwegverpackungen sind Verpackungen, die dazu konzipiert und bestimmt sind, nach dem Gebrauch mehrfach zum gleichen Zweck wiederverwendet zu werden und deren tatsächliche Rückgabe und Wiederverwendung durch eine ausreichende Logistik ermöglicht sowie durch geeignete Anreizsysteme, in der Regel durch ein Pfand, gefördert wird."
    -Einwegverpackung: "Einwegverpackungen sind Verpackungen, die keine Mehrwegverpackungen sind."
-Begriff: Inverkehrbringen:
     "Inverkehrbringen ist jede entgeltliche oder unentgeltliche Abgabe an Dritte im Geltungsbereich dieses Gesetzes mit dem Ziel des Vertriebs, des Verbrauchs oder der Verwendung. Nicht als Inverkehrbringen gilt die Abgabe von im Auftrag eines Dritten befüllten Verpackungen an diesen Dritten, wenn die Verpackung ausschließlich mit dem Namen oder der Marke des Dritten oder beidem gekennzeichnet ist."
    - Allgemeine Anforderungen an Verpackungen:
        -Verpackungen sind so zu entwickeln, herzustellen und zu vertreiben, dass
            1. Verpackungsvolumen und -masse auf das Mindestmaß begrenzt werden, das zur Gewährleistung der erforderlichen Sicherheit und Hygiene der zu verpackenden Ware und zu deren Akzeptanz durch den Verbraucher angemessen ist;
            2. ihre Wiederverwendung oder Verwertung, einschließlich des Recyclings, im Einklang mit der Abfallhierarchie möglich ist und die Umweltauswirkungen bei der Wiederverwendung, der Vorbereitung zur Wiederverwendung, dem Recycling, der sonstigen Verwertung oder der Beseitigung der Verpackungsabfälle auf ein Mindestmaß beschränkt bleiben;
            3. bei der Beseitigung von Verpackungen oder Verpackungsbestandteilen auftretende schädliche und gefährliche Stoffe und Materialien in Emissionen, Asche oder Sickerwasser auf ein Mindestmaß beschränkt bleiben;
            4. die Wiederverwendbarkeit von Verpackungen und der Anteil von sekundären Rohstoffen an der Verpackungsmasse auf ein möglichst hohes Maß gesteigert wird, welches unter Berücksichtigung der Gewährleistung der erforderlichen Sicherheit und Hygiene der zu verpackenden Ware und unter Berücksichtigung der Akzeptanz für den Verbraucher technisch möglich und wirtschaftlich zumutbar ist.
            %--> das als grundlage nehmen für die Anforderungen die allgemein im Unterschied zwischen einweg und mehrwegverpackungen den unterschied machen.
    -§ 5 Beschränkungen des Inverkehrbringens:
        -%als grenze und definition was man alles mit mehrwegverpackungen anstellen kann und was den unterschied zu einwegverpackungen bedeutet. Da diese beschränkungen sich auf gewisse Materialien/Stoffe beziehen muss hierbei noch die Wichtigkeit dieser Stoffe zu Verpackungen verschiedener Art und Weise verknüpft werden.
            -(1) Das Inverkehrbringen von Verpackungen oder Verpackungsbestandteilen, bei denen die Konzentration von Blei, Cadmium, Quecksilber und Chrom VI kumulativ den Wert von 100 Milligramm je Kilogramm überschreitet, ist verboten. Satz 1 gilt nicht für 
            1. Mehrwegverpackungen in eingerichteten Systemen zur Wiederverwendung, 
            2. Kunststoffkästen und -paletten, bei denen die Überschreitung des Grenzwertes nach Satz 1 allein auf den Einsatz von Sekundärrohstoffen zurückzuführen ist und die die in der Anlage 3 festgelegten Anforderungen erfüllen, 
            3. Verpackungen, die vollständig aus Bleikristallglas hergestellt sind, und
            4. aus sonstigem Glas hergestellte Verpackungen, bei denen die Konzentration von Blei, Cadmium, Quecksilber und Chrom VI kumulativ den Wert von 250 Milligramm je Kilogramm nicht überschreitet und bei deren Herstellung die in der Anlage 4 festgelegten Anforderungen erfüllt werden.


%(https://www.tuebingen.de/33361.html#/34231)
Der Bundesgesetzgeber hat im neuen Verpackungsgesetz beschlossen, dass ab dem Jahr 2023 !Restaurants, Imbisse und Cafés! beim Straßenverkauf zusätzlich zu Einwegverpackungen auch Mehrwegverpackungen anbieten müssen. Ausnahmen gelten für sogenannte „Kleinstbetriebe“ (unter 80 Quadratmetern Geschäftsfläche inklusive Lager und weniger als fünf Mitarbeiter:innen). Die Gerichte in Mehrwegverpackungen dürfen nicht teurer als in Einwegverpackungen sein. Bestimmte Einwegkunststoffprodukte im Sinne der Einwegkunststoffverbotsverordnung dürfen zudem nicht mehr in Verkehr gebracht werden.


%(https://packpart.eu/glossar/mehrwegverpackung/)
gesetzlichen Regelungen von Mehrwegverpackungen:
-Abfallvermeidungsprogramm der Bundesregierung sei 2013
--> Seitdem entwickelt das Bundesumweltministerium auf Grundlage aktueller Erkenntnisse neue Lösungen, Maßnahmen und Handlungsoptionen hinsichtlich der Reduzierung von Abfall.

-Seit dem 1. Januar gilt ein EU-weites Verbot für Exporte von schwer recycelbaren Kunststoffabfällen.


%% Generelles Vorgehen:
%% 1. erste Informationen über das Thema im generellen sammeln.
%% was finde ich wenn ich den titel eingebe?
%%    was finde ich wenn ich einwegverpackungen und mehrwegverpackungen eingebe?
%%    welche Art von content wird mir vorgeschlagen?
%%        also content welcher branche? lebensmitelindustrie, automobilindustrie
%%    beispiele für einwegverpackungen: karton, papier etc
%%    definitiopnen von einwegverpackungen in diversen branchen
%%    biologisch abbaubare verpackungen, recyclebare einwegverpackungen, umweltschutz, Bundesamt für abfallwirtschaft




------------------------------------------------------------------------------------------
        
        
%\subsection{Differenzierung von Subbaubranchen und die unterschiedliche Anwendbarkeit von Verpackungen}
%\label{sec:Grundlagenkapitel:Differenzierung von Subbaubranchen und die unterschiedliche Anwendbarkeit von Einweg- sowie Mehrwegverpackungen}


    %(https://de.wikipedia.org/wiki/Bauingenieurwesen)
 %   Keine genaue Differenzierung möglich, da die verschiedenen Branchen meist übergreifend sind und somit wird hier eine allgemeine, häufig vorkommende Variante gewählt und einzelne Fachgebiete addiert, die seltener zu finden sind, dennoch nicht zu vernachlässigbaren Einfluss auf die Anwendung von Mehrwegverpackungen hat.

  %  Eine grobe Unterteilung führt zu folgenden Subbaubranchen:
   %-Verkkehrbau
%    (Straßen- und Wegebau, Verkehrsplanung, Eisenbahnbau)
%    -Tiefbau
%    -Wasserbau  
%    (Wasserwirtschaft, Siedlungswasserwirtschaft, Abfallwirtschaft, Wasserbau, Küsteningenieurwesen, Energiewasserbau, Hydromechanik, Stahlwasserbau, Stauanlagenbau, Verkehrswasserbau, Hydrologie)



%(https://packpart.eu/glossar/mehrwegverpackung/)
gesetzlichen Regelungen von Mehrwegverpackungen:
-Abfallvermeidungsprogramm der Bundesregierung sei 2013
--> Seitdem entwickelt das Bundesumweltministerium auf Grundlage aktueller Erkenntnisse neue Lösungen, Maßnahmen und Handlungsoptionen hinsichtlich der Reduzierung von Abfall.

-Seit dem 1. Januar gilt ein EU-weites Verbot für Exporte von schwer recycelbaren Kunststoffabfällen.




%\section{Alternative Einwegverpackungen zu konventionellen Einwegverpackungen}
%\label{sec:aktueller Praxisstand von Verpackungen}

%Neben der Frage, ob und wann Mehrwegverpackungen sinnvoll sind einzusetzen, stellt sich ebenfalls die Option, zwar Einwegverpackungen einzusetzen, diese jedoch aus 

%        biologisch abbaubare Einwegverpackungen

%biologische einwegverpackungen
%synthetische einwegverpackungen

%und von beiden deren biologische Abbaufähigkeit 















